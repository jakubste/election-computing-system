\documentclass[pdflatex,11pt]{../aghdoc_version2}
% \documentclass{../aghdoc}               % przy kompilacji programem latex
\usepackage[polish]{babel}
\usepackage[utf8]{inputenc}

% dodatkowe pakiety
\usepackage[hidelinks]{hyperref}
\usepackage{enumerate}
\usepackage{caption}
\usepackage{listings}
\lstloadlanguages{TeX}

\lstset{
  literate={ą}{{\k{a}}}1
           {ć}{{\'c}}1
           {ę}{{\k{e}}}1
           {ó}{{\'o}}1
           {ń}{{\'n}}1
           {ł}{{\l{}}}1
           {ś}{{\'s}}1
           {ź}{{\'z}}1
           {ż}{{\.z}}1
           {Ą}{{\k{A}}}1
           {Ć}{{\'C}}1
           {Ę}{{\k{E}}}1
           {Ó}{{\'O}}1
           {Ń}{{\'N}}1
           {Ł}{{\L{}}}1
           {Ś}{{\'S}}1
           {Ź}{{\'Z}}1
           {Ż}{{\.Z}}1
}

%---------------------------------------------------------------------------

\author{Tomasz Kasprzyk, Daniel Ogiela, Jakub Stępak}
\shortauthor{T. Kasprzyk, D. Ogiela, J.Stępak}

\titlePL{System obliczający wyniki wyborów dla uogólnienia systemu k-Borda}

\shorttitlePL{System obliczający wyniki wyborów dla uogólnienia systemu k-Borda} % skrócona wersja tytułu jeśli jest bardzo długi

\thesistypePL{Dokumentacja procesowa}

\supervisorPL{dr hab. inż. Piotr Faliszewski}

\date{2016}

\departmentPL{Katedra Informatyki}

\facultyPL{Wydział Informatyki, Elektroniki i Telekomunikacji}

\setlength{\cftsecnumwidth}{10mm}

% umożliwienie żeby domyślnie dokument nie robił wcięć poza wybranymi (\indent w tym miejscu)

\newlength\tindent
\setlength{\tindent}{\parindent}
\setlength{\parindent}{0pt}
\renewcommand{\indent}{\hspace*{\tindent}}

% żeby stopki były zastosowane do stron gdzie rozpoczyna się rozdział
\usepackage{etoolbox}
\patchcmd{\chapter}{\thispagestyle{plain}}{\thispagestyle{fancy}}{}{}

\fancypagestyle{plain}{%
\fancyhf{} % clear all header and footer fields
\fancyhead[R]{\bfseries \thepage}
\fancyfoot[C]{System obliczający wyniki wyborów dla uogólnienia systemu k-Borda} % except the center
\renewcommand{\headrulewidth}{0.5pt}
\renewcommand{\footrulewidth}{0.5pt}}

%---------------------------------------------------------------------------

\begin{document}

\titlepages

\tableofcontents
\clearpage

%\chapter{Wprowadzenie}
\label{cha:wprowadzenie}

\LaTeX~jest systemem składu umożliwiającym tworzenie dowolnego typu dokumentów (w~szczególności naukowych i technicznych) o wysokiej jakości typograficznej (\cite{Dil00}, \cite{Lam92}). Wysoka jakość składu jest niezależna od rozmiaru dokumentu -- zaczynając od krótkich listów do bardzo grubych książek. \LaTeX~automatyzuje wiele prac związanych ze składaniem dokumentów np.: referencje, cytowania, generowanie spisów (treśli, rysunków, symboli itp.) itd.

\LaTeX~jest zestawem instrukcji umożliwiających autorom skład i wydruk ich prac na najwyższym poziomie typograficznym. Do formatowania dokumentu \LaTeX~stosuje \TeX a (wymiawamy 'tech' -- greckie litery $\tau$, $\epsilon$, $\chi$). Korzystając z~systemu składu \LaTeX~mamy za zadanie przygotować jedynie tekst źródłowy, cały ciężar składania, formatowania dokumentu przejmuje na siebie system.

%---------------------------------------------------------------------------

\section{Cele pracy}
\label{sec:celePracy}

Celem poniższej pracy jest zapoznanie studentów z systemem \LaTeX~w zakresie umożliwiającym im samodzielne, profesjonalne złożenie pracy dyplomowej w systemie \LaTeX.


%---------------------------------------------------------------------------

\section{Zawartość pracy}
\label{sec:zawartoscPracy}

W rodziale~\ref{cha:pierwszyDokument} przedstawiono podstawowe informacje dotyczące struktury dokumentów w \LaTeX u. Alvis~\cite{Alvis2011} jest językiem 



















%\chapter{Pierwszy dokument}
\label{cha:pierwszyDokument}

W rozdziale tym przedstawiono podstawowe informacje dotyczące struktury prostych plików \LaTeX a. Omówiono również metody kompilacji plików z zastosowaniem programów \emph{latex} oraz \emph{pdflatex}.

%---------------------------------------------------------------------------

\section{Struktura dokumentu}
\label{sec:strukturaDokumentu}

Plik \LaTeX owy jest plikiem tekstowym, który oprócz tekstu zawiera polecenia formatujące ten tekst (analogicznie do języka HTML). Plik składa się z dwóch części:
\begin{enumerate}%[1)]
\item Preambuły -- określającej klasę dokumentu oraz zawierającej m.in. polecenia dołączającej dodatkowe pakiety;

\item Części głównej -- zawierającej zasadniczą treść dokumentu.
\end{enumerate}


\begin{lstlisting}
\documentclass[a4paper,12pt]{article}      % preambuła
\usepackage[polish]{babel}
\usepackage[utf8]{inputenc}
\usepackage[T1]{fontenc}
\usepackage{times}

\begin{document}                           % część główna

\section{Sztuczne życie}

% treść
% ąśężźćńłóĘŚĄŻŹĆŃÓŁ

\end{document}
\end{lstlisting}

Nie ma żadnych przeciwskazań do tworzenia dokumentów w~\LaTeX u w~języku polskim. Plik źródłowy jest zwykłym plikiem tekstowym i~do jego przygotowania można użyć dowolnego edytora tekstów, a~polskie znaki wprowadzać używając prawego klawisza \texttt{Alt}. Jeżeli po kompilacji dokumentu polskie znaki nie są wyświetlane poprawnie, to na 95\% źle określono sposób kodowania znaków (należy zmienić opcje wykorzystywanych pakietów).


%---------------------------------------------------------------------------

\section{Kompilacja}
\label{sec:kompilacja}


Załóżmy, że przygotowany przez nas dokument zapisany jest w pliku \texttt{test.tex}. Kolejno wykonane poniższe polecenia (pod warunkiem, że w pierwszym przypadku nie wykryto błędów i kompilacja zakończyła się sukcesem) pozwalają uzyskać nasz dokument w formacie pdf:
\begin{lstlisting}
latex test.tex
dvips test.dvi -o test.ps
ps2pdf test.ps
\end{lstlisting}
%
lub za pomocą PDF\LaTeX:
\begin{lstlisting}
pdflatex test.tex
\end{lstlisting}

Przy pierwszej kompilacji po zmiane tekstu, dodaniu nowych etykiet itp., \LaTeX~tworzy sobie spis rozdziałów, obrazków, tabel itp., a dopiero przy następnej kompilacji korzysta z tych informacji.

W pierwszym przypadku rysunki powinny być przygotowane w~formacie eps, a~w~drugim w~formacie pdf. Ponadto, jeżeli używamy polecenia \texttt{pdflatex test.tex} można wstawiać grafikę bitową (np. w formacie jpg).



%---------------------------------------------------------------------------

\section{Narzędzia}
\label{sec:narzedzia}


Do przygotowania pliku źródłowego może zostać wykorzystany dowolny edytor tekstowy. Niektóre edytory, np. Emacs, mają wbudowane moduły ułatwiające składanie tekstów w LaTeXu (kolorowanie składni, skrypty kompilacji, itp.).

Jednym z bardziej znanych środowisk do składania dokumentów  \LaTeX a jest {\em Kile}. Aplikacja dostępna jest dla środowiska KDE począwszy od wersji 2. Zawiera edytor z podświetlaną składnią, zestawy poleceń \LaTeX a, zestawy symboli matematycznych, kreatory tabel, macierzy, skrypty kompilujące i konwertujące podpięte są do poleceń w menu aplikacji (i pasków narzędziowych), dostępne jest sprawdzanie pisowni, edytor obsługuje projekty (tzn. dokumenty składające się z~wielu plików), umożliwia przygotowanie i~zarządzanie bibliografią, itp.

Na stronie \underline{\texttt{http://kile.sourceforge.net/screenshots.php}} zamieszczono kilkanaście zrzutów ekranu środowiska {\em Kile}, które warto przejrzeć, by wstępnie zapoznać się z~możliwościami programu.

Bardzo dobrym środowiskiem jest również edytor gEdit z wtyczką obsługującą \LaTeX a. Jest to standardowy edytor środowiska Gnome. Po instalacji wtyczki obsługującej \LaTeX a, edytor nie ustępuje funkcjonalnościom środowisku Kile, a jest zdecydowanie szybszy w działaniu. Lista dostępnych wtyczek dla tego edytora znajduje się pod adresem \underline{\texttt{http://live.gnome.org/Gedit/Plugins}}. Inne polecane wtyczki to: 
\begin{itemize}
\item Edit shortcuts -- definiowanie własnych klawiszy skrótu;
\item Line Tools -- dodatkowe operacje na liniach tekstu;
\item Multi-edit -- możliwość jednoczesnej edycji w wielu miejscach tekstu;
\item Zoom -- zmiana wielkości czcionki edytora z użyciem rolki myszy;
\item Split View -- możliwość podziału okna edytora na 2 części. 
\end{itemize}



%---------------------------------------------------------------------------

\section{Przygotowanie dokumentu}
\label{sec:przygotowanieDokumentu}

Plik źródłowy \LaTeX a jest zwykłym plikiem tekstowym. Przygotowując plik
źródłowy warto wiedzieć o kilku szczegółach:

\begin{itemize}
\item
Poszczególne słowa oddzielamy spacjami, przy czym ilość spacji nie ma znaczenia.
Po kompilacji wielokrotne spacje i tak będą wyglądały jak pojedyncza spacja.
Aby uzyskać {\em twardą spację}, zamiast znaku spacji należy użyć znaku {\em
tyldy}.

\item
Znakiem końca akapitu jest pusta linia (ilość pusty linii nie ma znaczenia), a
nie znaki przejścia do nowej linii.

\item
\LaTeX~sam formatuje tekst. \textbf{Nie starajmy się go poprawiać}, chyba, że
naprawdę wiemy co robimy.
\end{itemize} 



%----------------------------------------------------------------------------
\chapter{Cele projektu}
\label{cha:cele_projektu}
Celem projektu jest stworzenie systemu obliczającego wyniki wyborów dla uogólnienia opisanego poniżej systemu k-Borda. 

Stworzona aplikacja webowa ma pozwalać użytkownikowi na szybkie definiowanie wyborów, jak również importowanie istniejących danych, w celu uzyskania wyników. 

Jako wybory rozumiemy nie tylko te, w których głosujący wyłaniają swoich przedstawicieli w organach władzy - swoje preferencje można także określić dla, na przykład, filmów jakie chcemy obejrzeć podczas seansu z przyjaciółmi, w jaką grę chcielibyśmy zagrać, jaką restaurację wybrać na rodzinne spotkanie itp.

Ustalenie preferencji to jedno, ale o tym, kto lub co zostanie wybrane na podstawie preferencji wszystkich głosujących, decyduje również system wyborczy.
Określając parametr $p$ systemu wyborczego, aplikacja ma za zadanie ilustrować jak, w zależności od tego parametru, zmieniają się wyniki wyborów przy tym samym zestawie preferencji.

W pierwszej kolejności użytkownik określa listę preferencji wyborców. Określając swoje preferencje, 
każdy wyborca porządkuje kandydatów w kolejności od najlepszego do najgorszego. Lista preferencji może 
zostać zaimportowana z odpowiednio sformatowanego pliku, wygenerowana z rozkładu normalnego lub wygenerowana z zaznaczonych 
punktów na płaszczyźnie.
Na wprowadzonych do systemu listach preferencji i ustalonego rozmiaru komitetu, użytkownik może 
wielokrotnie uruchamiać algorytm wyłaniania zwycięzców z różnym parametrem.

\chapter{Studium wykonalności}
\label{cha:studium_wykonalnosci}

\section{Opis stanu istniejącego}
\label{sec:opis_stanu_istniejacego}
Tworzony system nie posiada swojego pierwowzoru.

\section{Opis wymagań}
\subsection{Wymagania funkcjonalne}
Ponieważ zastosowano ewolucyjny proces tworzenia oprogramowania, wymagania
funkcjonalne zbierane były na bieżąco w trakcie trwania projektu. Dla kolejnych wersji
systemu dodawano nowe wymagania, bądź nieco zmieniano już istniejące. Wymagania były
dodawane lub zmienianie na bazie rozmów z klientem i rozwoju prac. Poniżej przedstawiono
wymagania dla kolejnych wersji systemu oraz wymagania przedstawione na początku przez
klienta. \\ \\
\textbf{Wymagania przedstawione przez klienta na pierwszym spotkaniu:}
\begin{itemize}
\item Obliczanie wyników wyborów przez zaimplementowane w tym celu algorytmy
heurystyczne. Definicja wyborów została ściśle określona.
\item Przystępne przedstawienie wyników wyborów
\end{itemize}
\vspace{\baselineskip}
\textbf{Wymagania dla 1. wersji systemu:}
\begin{itemize}

\item Wczytanie wyborów do systemu z pliku formatu $.soc$
\item Obliczanie normy $\ell_p$
\item Obliczanie wyników wyborów przez algorytm typu brute-force
\end{itemize}
\vspace{\baselineskip}
\textbf{Wymagania dla 2. wersji systemu:}
\begin{itemize}
\item Przeniesienie wcześniej dodanych funkcjonalności do aplikacji webowej
\item Generowanie wyborów z rozkładu normalnego
\item Logowanie użytkowników na swoje konto
\item Tworzenie i usuwanie wyborów
\item Wyświetlanie wyborów i ich wyników dla wyborów wygenerowanych z rozkładu
normalnego
\end{itemize}
\vspace{\baselineskip}
\textbf{Wymagania dla 3. wersji systemu:}
\begin{itemize}
\item Obliczanie wyników wyborów algorytmem zachłannym
\item Obliczanie wyników wyborów algorytmem genetycznym
\end{itemize}
\vspace{\baselineskip}
\textbf{Wymagania dla 4. – ostatecznej – wersji systemu:}
\begin{itemize}
\item Wydajniejsze obliczanie wyników wyborów przez algorytm zachłanny i genetyczny
\item Możliwość parametryzowania z poziomu użytkownika algorytmu genetycznego
\item Obliczanie wyników wyborów za pomocą algorytmu zachłannego
według zasady \mbox{\textit{Chamberlina-Couranta}}
\item Generowanie wyborów z punktów zaznaczonych na płaszczyźnie przez użytkownika
\item Poprawienie użyteczności systemu
\end{itemize}

\subsection{Wymagania niefunkcjonalne}
Wymagania niefunkcjonalne nie zmieniały się zbytnio w trakcie rozwoju projektu.

\subsubsection{Wymagania produktowe}
\begin{itemize}
\item Szybkość i dokładność wykonywanych obliczeń - obliczanie wyników wyborów w
systemie $k-Borda$ jest czasochłonne. Zadaniem projektowym jest opracowanie
algorytmów, które pozwolą na szybsze otrzymywanie wyników wyborów kosztem ich
przybliżenia. Stworzone algorytmy powinny być użyteczne dla możliwie dużych
danych.
\end{itemize}

\subsubsection{Wymagania organizacyjne}
\begin{itemize}
\item Dotrzymanie terminów na przedstawienie poszczególnych elementów pracy -
realizacja projektu inżynierskiego odbywa się według trybu ustalonego przez władze
uczelni. Pierwszym aspektem są specjalnie wydzielone przedmioty - \textit{Pracownia
Projektowa 1} i \textit{Pracownia Projektowa 2} - które wspomagają i motywują studentów
do wykonywania poszczególnych części projektu w wyznaczonym czasie. Na
zajęciach seminaryjnych przedstawiano w określonych terminach kolejno: \textit{Wizję
produktu}, \textit{Studium wykonalności} i prezentację kolejnych prototypów. Wyznaczono
również termin na oddanie dokumentacji i przedstawienie produktu.
\end{itemize}

\newpage
\section{Strategia testowania}
Poprawność działania podstawowych funkcjonalności zapewnianych przez system
gwarantują testy jednostkowe. Pisane były one na bieżąco po dodawaniu kolejnych
możliwości i udogodnień systemu.

\indent W celu zagwarantowania poprawności działania wcześniej dodanych funkcjonalności, po
wykonaniu i wdrożeniu nowych funkcjonalności systemu, skorzystano z praktyki ciągłej
integracji. Zrealizowano ją za pomocą serwisu \textit{Travis CI}. Serwis zapewnia automatyczne
wykonanie przygotowanych testów jednostkowych, po każdej wprowadzonej zmianie do
repozytorium kodu źródłowego
\section{Aspekt technologiczny}
\label{sec:apekt_technologiczny}
Do stworzenia oprogramowania wykorzystano język \textit{Python 2.7}. System wykonany jest w
formie aplikacji internetowej. Do zrealizowania aplikacji webowej wykorzystano framework
dla języka programowania \textit{Python} - \textit{Django 1.9}. \textit{Django} dostarcza w pełni funkcjonalny
system uwierzytelniania, zapewnia obsługę kont, grup oraz uprawnień użytkowników.
Oprócz zaoszczędzenia czasu na realizację powyższych, nieistotnych dla realizacji tematu
projektu, funkcjonalności systemu, głównym powodem wybrania tego stosu technologicznego było doświadczenie w
tworzeniu aplikacji z użyciem \textit{Django} posiadane przez jednego z członków zespołu
realizującego niniejszy projekt.

\chapter{Analiza ryzyka}
\section{Identyfikacja zagrożeń}
Spośród wielu czynników mogących mieć wpływ na przedłużanie się czasu wykonania
poszczególnych elementów systemu za najważniejsze można uznać:
\begin{enumerate}
\item Problem ze stworzeniem satysfakcjonującego algorytmu heurystycznego.
Zaprojektowanie i implementacja algorytmów heurystycznych to najważniejsze i
największe wymaganie dla tego projektu. W związku z tym, realizację tego
wymagania może utrudnić wiele nieprzewidzianych czynników.
\item Potencjalny problem z czasem trwania obliczania normy $\ell_p$ dla dużych wartości parametru $p$ . Jest to jeden
z czynników mogących wpłynąć na stworzenie wydajnego algorytmu
heurystycznego. Jeżeli ten problem wystąpi, utrudni to stworzenie
satysfakcjonującego algorytmu heurystycznego.
\item Koordynacja prac zespołu. Praca w trzyosobowej grupie może nieść ze sobą zarówno
korzyści jak i problemy. Konieczny jest wyraźny podział zadań - pomocne okażą się
narzędzia do kontrolowania podziału pracy. \\
Należy liczyć się również z problemami z synchronizacją czasową. Różne plany
zajęć członków zespołu (różne przedmioty obieralne i związane z tym obciążenie
czasowe w danej części semestru), praca zawodowa członków zespołu, konieczność
odbycia staży wakacyjnych, pobyt poza Krakowem (brak możliwości spotkania się
we trójkę) mogą utrudnić tworzenie systemu.
\end{enumerate}

\section{Analiza zagrożeń}
{
\centering
\begin{tabular}{|p{5cm}|c|c|p{5cm}|}
\hline 
\textbf{Zagrożenie} & \textbf{Prawdopodobieństwo} & \textbf{Konsekwencje} & \textbf{Strategia} \\  
\hline 
Problem ze
stworzeniem
satysfakcjonującego
algorytmu
heurystycznego & Duże & Poważne & Szybkie stworzenie
bazy systemu i
skupienie się na
głównym zadaniu
projektu \\ 
\hline 
Czas trwania
obliczania normy $\ell_p$
dla dużych $p$ & Duże & Poważne & Zaoszczędzenie
czasu działania
algorytmu poprzez
minimalizację
operacji na bazie
danych \\
\hline 
Koordynacja prac zespołu
- problem z
przydziałem zadań i
czasem ich
wykonania & Duże & Znośne & Zastosowanie
systemu przydziału
zadań i częste
wspólne wewnętrzne spotkania \\ 
\hline 
\end{tabular}
\captionof{table}{Analiza zagrożeń}
} 

\chapter{Przyjęta metodyka pracy}
\section{Tworzenie oprogramowania}
Proces tworzenia oprogramowania dla tego projektu inżynierskiego posiadał więcej lub mniej
cech wielu metod tworzenia oprogramowania. Najwięcej wspólnego miał z ewolucyjnym
procesem tworzenia oprogramowania. Można tak uznać głównie ze względu na początkowo
ogólne, niedoprecyzowane wymagania klienta. Właściwie jedynym wymaganiem dla
systemu było stworzenie algorytmów heurystycznych do obliczania wyborów zdefiniowanych
według ściśle określonych zasad. Cała otoczka systemu pozostała w gestii członków
zespołu. W trakcie kolejnych etapów projektu często kontaktowano się z klientem i
wymagania doprecyzowywano. 

\indent Wątpliwość co do pełnego uznania tworzenia tego systemu
jako ewolucyjnego procesu tworzenia oprogramowania, może pojawić się przy przejściach
między kolejnymi wersjami systemu. Opisane w dalszej części dokumentacji przejścia, miały niekiedy cechy przyrostów w przyrostowym modelu tworzenia oprogramowania. Ze względu na klarowność opisu przebiegu prac i zgodność przytłaczającej liczby cech, zdecydowano uznać tworzenie tego systemu jako proces
ewolucyjnego tworzenia oprogramowania.

\indent Wykorzystano tworzenie badawcze, które polega na częstym kontakcie z klientem w celu
ciągłego badania i weryfikowania wymagań systemu. Opracowano pierwotną wersję
systemu, a następnie udoskonalano go w wielu wersjach, aż do uzyskania ostatecznej
wersji. Każda kolejna wersja systemu dodawała nową funkcjonalność, bądź przyspieszała
działanie wcześniej zrealizowanych funkcjonalności.

\section{Podział prac}
Wykonane czynności przez poszczególnych członków zespołu: \\ \\
Tomasz Kasprzyk:
\begin{itemize}
\item generacja wyborów z rozkładu normalnego
\item wizualizacja wyników wyborów i wydajności algorytmów na wykresach
\item tworzenie wyborów poprzez zaznaczanie punktów na płaszczyźnie
\item współtworzenie dokumentacji projektowej
\item współtworzenie testów jednostkowych i porównawczych
\item współtworzenie prezentacji na zajęcia seminaryjne
\end{itemize}
\newpage
Daniel Ogiela:
\begin{itemize}
\item projekt i implementacja algorytmu zachłannego zależnego od parametru $p$
\item implementacja algorytmu zachłannego według zasady \textit{Chamberlina-Couranta}
\item współtworzenie dokumentacji projektowej
\item współtworzenie testów jednostkowych i porównawczych
\item współtworzenie prezentacji na zajęcia seminaryjne
\end{itemize}

Jakub Stępak:
\begin{itemize}
\item stworzenie szkieletu aplikacji
\item konfiguracja ciągłej integracji
\item implementacja algorytmu \textit{brute-force}
\item implementacja algorytmu genetycznego
\item przedstawienie wyników na wykresie i tabelaryczne
\item współtworzenie testów jednostkowych i porównawczych
\item współtworzenie dokumentacji projektowej
\item współtworzenie prezentacji na zajęcia seminaryjne
\end{itemize}

\section{Komunikacja z managerem i klientem}
W trakcie prac nad systemem odbywały się spotkania z managerem i klientem. Spotkania z
managerem miały formę zajęć seminaryjnych, na których przedstawiano kolejne prezentacje
dotyczące pracy inżynierskiej. Przedstawiono wstępną wizję projektu, studium wykonalności
oraz prototyp systemu. Oprócz wymienionych artefaktów na zajęciach z managerem
regularnie zdawano raport z aktualnego postępu prac.

Spotkania z klientem odbywały się rzadziej i miały formę sprawozdań z aktualnego stanu
prac. Klient na podstawie sprawozdań sugerował kolejne czynności i ulepszenia systemu,
nad którymi powinni się skupić członkowie zespołu inżynierskiego.

\newpage
\section{Wykorzystane narzędzia do zarządzania projektem}
Do zarządzania wykonywanymi w ramach projektu zadaniami wykorzystano aplikację
internetową \textit{Trello}. Narzędzie służyło zarówno do określania zadań, przydzielania ich do
konkretnych członków zespołu, jak i do gromadzenia istotnych dla realizacji projektu
informacji. Narzędzie umożliwiało określenie typu zadania oraz wskazanie, na którym etapie
wykonywania zadania znajduje się dany członek zespołu.

Do przechowywania kodu źródłowego oprogramowania wykorzystano serwis internetowy
\textit{GitHub}. Repozytorium jest dostępne publicznie.

Dostęp do repozytorium kodu źródłowego i \textit{Trello} na każdym etapie procesu tworzenia
oprogramowania miał zarówno manager i klient. 

W celu zdalnego kontaktowania się pomiędzy członkami zespołu korzystano z
komunikatora \textit{Slack}. Komunikator ten, oprócz podstawowej funkcjonalności prowadzenia
konwersacji, pozwala również na transfer plików, przesyłanie fragmentów sformatowanego
kodu w postaci $code \ snippets$ oraz wyszukiwanie powyższych w całej historii konwersacji.
Korzystano również z mechanizmu botów, w celu monitorowania zmian w repozytorium czy ciągłej integracji.

\textit{Pycharm}, \textit{IDE} firmy \textit{JetBrains}. W wersji dla studentów udostępniony jest pakiet
zaawansowanych funkcjonalności, w szczególności integracja z frameworkiem \textit{Django}. Z
poziomu \textit{IDE} można również aktualizować repozytorium (opcje dostępne w zakładce \textit{VCS}).
W celu przepisywania dokumentacji z \textit{Google Docs} do \textit{LaTeX} korzystano z edytora
\textit{TexMaker}. Funkcja \textit{WYSIWYG} znacznie przyspiesza proces tworzenia poprawnie
sformatowanego dokumentu.

Serwisy \textit{Travis CI} oraz \textit{Coveralls.io} służyły automatycznemu uruchamianiu testów jednostkowych wraz z badaniem pokrycia kodu testami. Serwisy te połączono dla wygody ze \textit{Slackiem} zespołu projektowego poprzez wykorzystanie wspomnianych wyżej botów.

Dodatkowo każda zmiana na głównej gałęzi repozytorium skutkuje automatycznym zbudowaniem serwisu na platformie \textit{Heroku} - aktualna wersja serwisu jest zawsze dostępna pod adresem \\  \url{https://election-computing-system.herokuapp.com}.

\section{Weryfikacja wyników projektu}
Jakość obliczanych wyników wyborów weryfikowano za pomocą testów
porównawczych oraz oceny wizualnej. Testy porównawcze polegały na porównywaniu wyników generowanych przez różne
algorytmy heurystyczne. Porównanie dotyczyło czasów działania algorytmów oraz zadowoleń wyborców z wyniku wypracowanych przez algorytmy. Dodatkowo dla małych rozmiarów danych porównywano wyniki algorytmów heurystycznych z wynikami uzyskanymi metodą \textit{brute-force}.

Ocena wizualna dotyczyła ocenienia wykresów wyników wyborów – sprawdzano czy w miarę wzrostu parametru $p$, punkty reprezentujące zwycięskich kandydatów stopniowo rozpraszały się na mapie spektrum poglądów, co byłoby zgodne z oczekiwaniami dotyczącymi zachowywania się wyników wyborów w tym systemie 
wyborczym. Ocena wizualna dotyczyła tylko wyborów generowanych z rozkładu normalnego. Dla tych 
wyborów, wyborcy i kandydaci są reprezentowani jako punkty na płaszczyźnie.

\chapter{Przebieg prac}
\section{Harmonogram}
Ze względu na ewolucyjny charakter tworzenia oprogramowania, szczegółowe zadania były wyznaczane na bieżąco. Dostosowywano je do aktualnych wymagań. Poniżej przedstawiono orientacyjny harmonogram określony na początku pracy, który
przewidywał czas ukończenia najważniejszych elementów systemu.
\begin{itemize}
\item Maj 2016
	\begin{itemize}
	\item Implementacja algorytmu typu brute-force
	\item Praca nad algorytmem pozwalającym na szybsze wykonywanie obliczeń
	\end{itemize}
\item Lipiec 2016
	\begin{itemize}
	\item Implementacja pierwszych wersji algorytmów heurystycznych
	\item Stworzenie interfejsu
	\item Importowanie danych z \textit{PrefLib}
	\item Symulowanie wyborów z zastosowaniem rozkładu normalnego
	\end{itemize}
\item Październik 2016
	\begin{itemize}
	\item Przyspieszenie algorytmów heurystycznych
	\end{itemize}
\item Listopad 2016
	\begin{itemize}
	\item Przygotowanie dokumentacji
	\item Udoskonalenie interfejsu
	\end{itemize}
\end{itemize}

\newpage
\section{Opis zebrania podstawowych wymagań klienta i podjęcie
decyzji projektowych}

\subsection{Czas trwania}
Marzec 2016
\subsection{Zadania wyznaczone w tej fazie}
Głównym zadaniem w tej fazie procesu tworzenia oprogramowania było zebranie
podstawowych wymagań klienta i ich dokładne przeanalizowanie. Na bazie specyfikacji
wymagań należało wybrać odpowiednie technologie do realizacji produktu.
\subsection{Opis przebiegu prac}
W celu wykonania wyżej wymienionych zadań odbyto spotkanie z przyszłym klientem. Na
spotkaniu poruszono zagadnienia dotyczące obszaru zainteresowań klienta, które były
bezpośrednio związane z tematem projektu inżynierskiego. Klient przekazał podstawowe wymagania odnośnie przyszłego systemu. Na bazie uzyskanych informacji i materiałów
przystąpiono do dokładnego przestudiowania dziedziny problemu. Po wnikliwej analizie
podjęto główne decyzje projektowe. Przygotowano również prezentację na zajęcia z
managerem opisującą zadany projekt inżynierski.

\subsection{Wynik}
\begin{itemize}
\item Lista początkowych wymagań funkcjonalnych i niefunkcjonalnych systemu
\item Wybór technologii \textit{Python 2.7}
\item Opis projektu inżynierskiego w postaci prezentacji \textit{Wizji produktu}
\end{itemize}

\newpage
\section{Opis kolejnych wersji systemu}
\subsection{Wersja 1.}
\subsubsection{Czas trwania}
kwiecień 2016
\subsubsection{Cele wyznaczone do osiągnięcia}
Jednym z celów do osiągnięcia w pierwszej wersji systemu (niespełniającej wszystkich
wymagań) było zaimplementowanie wczytywania do systemu preferencji wyborców
otrzymanych z pliku odpowiedniego formatu. Drugim ważnym celem była implementacja
algorytmu typu \textit{brute-force} do obliczania wyników wyborów. Implementacja wymienionych
zadań poza dodaniem funkcjonalności systemowi miała posłużyć dookreśleniu wymagań
systemu i poznaniu specyfiki rozwiązywanego problemu. Algorytm typu \textit{brute-force} miał
również przydać się w późniejszej fazie projektu do testów porównawczych docelowego
algorytmu dla małych rozmiarów danych wejściowych.
\subsubsection{Opis przebiegu prac}
Przed przystąpieniem do implementacji poświęcono czas na zaprojektowanie struktury obiektowej informacji przechowywanych w systemie. Prace rozpoczęto od implementacji wczytywania wyborów 
\mbox{z pliku} z rozszerzeniem $.soc$.
Następnie dodano liczenie normy $\ell_p$ oraz algorytmu typu brute-force pozwalającego na
obliczanie wyników wyborów dla prostych danych. Z czasem stwierdzono, że wczytywanie
danych z pliku jest niewystarczające dla testowania dużych danych, stąd rozpoczęto prace
nad generowaniem wyborów z rozkładu normalnego. Zdecydowano o możliwości
przedstawiania wyborców \mbox{i kandydatów} jako punkty na płaszczyźnie. Preferencje wyborcze
danego wyborcy zdecydowano ustalać na podstawie odległości euklidesowych punktu
przedstawiającego tego wyborcę do punktów przyporządkowanych kandydatom do komitetu.
W dalszych pracach dodano również walidację poprawności pliku wejściowego. 
\subsubsection{Wynik}
\begin{itemize}
\item Diagram klas
\item Pierwsza wersja systemu uruchamiana z poziomu konsoli \textit{Pythona} przyjmująca za
argumenty parametr $p$ konieczny do obliczania normy $\ell_p$ oraz plik z rozszerzeniem
$.soc$.
\item Dodane funkcjonalności: wczytywanie wyborów z pliku z rozszerzeniem $.soc$,
algorytm typu brute-force, walidacja danych z pliku wejściowego
\item Rozpoczęcie prac nad generowaniem własnych wyborów
\item Podjęcie decyzji o wykonaniu systemu w postaci aplikacji webowej i wykorzystaniu
do tego frameworka \textit{Django 1.9}
\end{itemize}

\newpage
\subsection{Wersja 2.}
\subsubsection{Czas trwania}
maj 2016
\subsubsection{Cele wyznaczone do osiągnięcia}
Głównym celem dla tej wersji systemu było przeniesienie zrealizowanych wcześniej
funkcjonalności do stworzonej na bazie frameworka \textit{Django 1.9} aplikacji webowej. Celem
było również dokończenie prac nad generacją własnych wyborów, w szczególności dodanie
możliwości generowania preferencji w obiektowo zorientowany sposób.
\subsubsection{Opis przebiegu prac}
Prace rozpoczęto od stworzenia szkieletu aplikacji webowej we frameworku \textit{Django 1.9}. 
Następnie dodawano kolejne funkcjonalności systemu: możliwość rejestracji i
logowania się użytkowników, tworzenie wyborów, usuwanie wyborów czy wyświetlanie dla
danego użytkownika stworzonych wyborów. Stworzono modele pozwalające na zapisanie
obiektów w bazie danych. Równolegle dodano możliwość generowania wyborów, które
odzwierciedlały obiektową strukturę danych. Na bazie tego i stworzonych funkcjonalności w
poprzedniej wersji systemu dodano do aplikacji webowej możliwości: wczytywania wyborów
z pliku, walidację danych z pliku wejściowego, generowanie wyborów z rozkładu normalnego
oraz wyświetlanie preferencji wszystkich wyborców. Do kodu w repozytorium dodano ciągłą
integrację za pomocą narzędzia \textit{Travis CI}. W dalszej kolejności zajęto się umożliwieniem w aplikacji webowej obliczania wyników wyborów algorytmem typu brute-force. Równolegle
rozpoczęto pracę nad wyświetlaniem wyborców i kandydatów reprezentowanych jako punkty
na wykresie $2D$ (dla wyborów generowanych z rozkładu normalnego). Po ukończeniu tych
zadań dodano funkcjonalność wyświetlania wyników wyborów na wykresie $2D$. W
międzyczasie skonfigurowano wdrożenie systemu na platformie \textit{Heroku}. W ramach zajęć
seminaryjnych wykonano studium wykonalności, w którym podjęto szereg decyzji
projektowych.
\subsubsection{Wynik}
\begin{itemize}
\item Aplikacja webowa wdrożona na platformę \textit{Heroku}
\item Przeniesione funkcjonalności z poprzedniej wersji systemu do aplikacji webowej
\item Dodane nowe funkcjonalności: rejestracja i logowanie użytkowników, generacja
wyborów z rozkładu normalnego, tworzenie oraz usuwanie wyborów, wyświetlanie
dla danego użytkownika stworzonych wyborów, wyświetlanie kandydatów i wyborców
reprezentowanych jako punkty oraz wyników wyborów na wykresach $2D$ (dla
wyborów generowanych z rozkładu normalnego)
\item Studium wykonalności
\end{itemize}
\newpage

\subsection{Wersja 3.}
\subsubsection{Czas trwania}
koniec maja 2016 - czerwiec 2016
\subsubsection{Cele wyznaczone do osiągnięcia}
Głównym celem wyznaczonym dla tej wersji systemu było stworzenie pierwszych wersji
dwóch algorytmów heurystycznych – genetycznego oraz zachłannego. 
Ponadto zdecydowano się skupić nad optymalizacją liczby operacji odczytu danych z bazy danych, jako czynnika, 
który znacząco spowalniał wykonanie programu.
\subsubsection{Opis przebiegu pracy}
Rozpoczęto od ograniczenia liczby operacji odczytu preferencji wyborczych z bazy danych
przy wykonywaniu algorytmu \textit{brute-force}. Zredukowano liczbę tych operacji poprzez
jednokrotne wczytanie z bazy danych preferencji wyborczych i przechowywanie ich w pamięci podręcznej.
Po wykonaniu zadania redukcji operacji odczytu danych z bazy danych,
przystąpiono do równoległej realizacji algorytmu genetycznego i algorytmu zachłannego. W
ramach zajęć seminaryjnych wykonano prezentację, w której pokazano funkcjonalność
aktualnej wersji systemu.
\subsubsection{Wynik}
\begin{itemize}
\item Dodanie funkcjonalności obliczania wyników wyborów za pomocą algorytmu
genetycznego i zachłannego
\item Czas działania algorytmu zachłannego wyraźnie krótszy od algorytmu typu
brute-force
\item Działający algorytm genetyczny, z parametrami zapisanymi na stałe w kodzie programu
\item Redukcja operacji odczytu danych z bazy danych przy obliczaniu wyników wyborów
algorytmem typu brute-force
\end{itemize}

\newpage
\subsection{Wersja 4. – końcowa}
\subsubsection{Czas trwania}
Sierpień 2016 - Grudzień 2016
\subsubsection{Cele wyznaczone do osiągnięcia}
Najważniejszym celem dla tej wersji systemu jest maksymalne przyspieszenie działania
algorytmu genetycznego i zachłannego.
Kolejnym zadaniem jest ułatwienie porównywania rezultatów obliczeń (w szczególności:
czasu wykonywania obliczeń oraz wyniku zwycięskiego komitetu) dla różnych algorytmów.
Ponadto skupiono się na ulepszeniu dynamicznego wyświetlania wyników wyborów i dodaniem 
tworzenia wyborów przez zaznaczanie punktów na płaszczyźnie.
\subsubsection{Opis przebiegu pracy}
Algorytmy heurystyczne przyspieszano głównie poprzez eliminację niepotrzebnych operacji
na bazie danych. Zmniejszyło to czas dostępu do danych i tym samym czas trwania
algorytmów. 

Wprowadzono możliwości ustalania parametrów algorytmu genetycznego (takich jak: liczba cykli, prawdopodobieństwo
mutacji, część puli podlegająca krzyżowaniu) z poziomu interfejsu webowego.
Dostosowano wygląd listy rezultatów, tak by pokazywała zastosowane parametry.

Dla realizacji celu ulepszenia sposobu wyświetlania wyników zaimplementowano możliwość porównywania 
wielu wyników na jednym wykresie, przełączając się między wynikami poprzez suwak. 
Dodano również wykres zestawiający czasy obliczeń algorytmów w zależności od parametru $p$.

Tworzenie wyborów poprzez zaznaczanie punktów na płaszczyźnie wiązało się z dodaniem graficznego 
interfejsu do zaznaczania punktów oraz przekazaniem tak uzyskanych danych do bazy.

\subsubsection{Wynik}
\begin{itemize}
\item Dodanie wersji algorytmu zachłannego dla szczególnego przypadku - algorytm $Greedy \ CC$
\item Możliwość porównania zmieniających się wyników na jednym wykresie.
\item Dodanie informacji o czasie wykonania, wyniku punktowym i ew. parametrach algorytmu na liście rezultatów
\item Dodanie wykresu porównującego czasy obliczeń poszczególnych algorytmów w zależności od parametru $p$
\item Dodanie możliwości tworzenia wyborów poprzez zaznaczanie punktów na płaszczyźnie
\item Dodanie możliwości usuwania pojedynczego rezultatu
\end{itemize}

\section{Problemy napotkane w czasie realizacji projektu}
Jednym z napotkanych problemów podczas realizacji systemu był niezadowalający czas
liczenia normy $\ell_p$ . W związku z tym należało znaleźć miejsca w algorytmie, które
zaoszczędzałyby czas trwania obliczeń. Zlokalizowano również miejsca, w których program
niepotrzebnie wiele razy pobierał dane z bazy danych – w tych miejscach dane zostały pobrane wcześniej do pamięci podręcznej co znacząco ograniczyło czas potrzebny na operacje odczytu z bazy danych.
% itd.
% \appendix
% \include{dodatekA}
% \include{dodatekB}
% itd.

\bibliographystyle{alpha}
\bibliography{bibliografia}
%\begin{thebibliography}{1}
%
%\bibitem{Dil00}
%A.~Diller.
%\newblock {\em LaTeX wiersz po wierszu}.
%\newblock Wydawnictwo Helion, Gliwice, 2000.
%
%\bibitem{Lam92}
%L.~Lamport.
%\newblock {\em LaTeX system przygotowywania dokumentów}.
%\newblock Wydawnictwo Ariel, Krakow, 1992.
%
%\bibitem{Alvis2011}
%M.~Szpyrka.
%\newblock {\em {On Line Alvis Manual}}.
%\newblock AGH University of Science and Technology, 2011.cccccc
%\newblock \\\texttt{http://fm.ia.agh.edu.pl/alvis:manual}.
%
%\end{thebibliography}

\end{document}
