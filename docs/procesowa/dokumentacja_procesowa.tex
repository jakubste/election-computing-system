\documentclass[pdflatex,11pt]{aghproc}
% \documentclass{aghdpl}               % przy kompilacji programem latex
\usepackage[polish]{babel}
\usepackage[utf8]{inputenc}

% dodatkowe pakiety
\usepackage{enumerate}
\usepackage{listings}
\lstloadlanguages{TeX}

\lstset{
  literate={ą}{{\k{a}}}1
           {ć}{{\'c}}1
           {ę}{{\k{e}}}1
           {ó}{{\'o}}1
           {ń}{{\'n}}1
           {ł}{{\l{}}}1
           {ś}{{\'s}}1
           {ź}{{\'z}}1
           {ż}{{\.z}}1
           {Ą}{{\k{A}}}1
           {Ć}{{\'C}}1
           {Ę}{{\k{E}}}1
           {Ó}{{\'O}}1
           {Ń}{{\'N}}1
           {Ł}{{\L{}}}1
           {Ś}{{\'S}}1
           {Ź}{{\'Z}}1
           {Ż}{{\.Z}}1
}

%---------------------------------------------------------------------------

\author{Tomasz Kasprzyk, Daniel Ogiela, Jakub Stępak}
\shortauthor{T. Kasprzyk, D. Ogiela, J.Stępak}

\titlePL{System obliczający wyniki wyborów dla uogólnienia systemu k-Borda}

\shorttitlePL{Przygotowanie pracy dyplomowej w~systemie \LaTeX} % skrócona wersja tytułu jeśli jest bardzo długi

\thesistypePL{Dokumentacja procesowa}

\supervisorPL{dr hab. inż. Piotr Faliszewski}

\date{2016}

\departmentPL{Katedra Informatyki}

\facultyPL{Wydział Informatyki, Automatyki, Informatyki i Elektroniki}

\setlength{\cftsecnumwidth}{10mm}

%---------------------------------------------------------------------------

\begin{document}

\titlepages

\tableofcontents
\clearpage

%\include{rozdzial1}
%\include{rozdzial2}
%----------------------------------------------------------------------------
\chapter{Cele projektu}
\label{cha:cele_projektu}

\chapter{Studium wykonalności}
\label{cha:studium_wykonalnosci}

\section{Opis stanu istniejącego}
\label{sec:opis_stanu_istniejacego}

\section{Wymagania dla systemu}
\label{sec:wymagania_dla_systemu}

\subsection{Metoda obliczania wyników wyborów}
\label{subsec:metoda_obliczania_wynikow_wyborow}

\subsubsection{Metoda Bordy}
\label{subsubsec:metoda_bordy}

\subsubsection{Metoda k-Borda}
\label{subsubsec:metoda_k_borda}

\subsubsection{Uogólnienie - system $\ell_p$ Borda}
\label{subsubsec:system_ell_p_borda}

\paragraph{Norma $\ell_p$}
\label{par:norma_ell_p}

\subsection{Format danych wejściowych}
\label{subsec:format_danych_wejsciowych}

\subsection{Szybkość i dokładność wykonywanych obliczeń}
\label{subsec:szybkosc_i_dokladnosc_wykonywanych_obliczen}

\section{Wybór frameworku dla aplikacji webowej}
\label{sec:wybor_frameworku_dla_aplikacji_webowej}

% itd.
% \appendix
% \include{dodatekA}
% \include{dodatekB}
% itd.

\bibliographystyle{alpha}
\bibliography{bibliografia}
%\begin{thebibliography}{1}
%
%\bibitem{Dil00}
%A.~Diller.
%\newblock {\em LaTeX wiersz po wierszu}.
%\newblock Wydawnictwo Helion, Gliwice, 2000.
%
%\bibitem{Lam92}
%L.~Lamport.
%\newblock {\em LaTeX system przygotowywania dokumentów}.
%\newblock Wydawnictwo Ariel, Krakow, 1992.
%
%\bibitem{Alvis2011}
%M.~Szpyrka.
%\newblock {\em {On Line Alvis Manual}}.
%\newblock AGH University of Science and Technology, 2011.cccccc
%\newblock \\\texttt{http://fm.ia.agh.edu.pl/alvis:manual}.
%
%\end{thebibliography}

\end{document}
