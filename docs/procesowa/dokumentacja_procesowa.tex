\documentclass[pdflatex,11pt]{../aghdoc}
% \documentclass{../aghdoc}               % przy kompilacji programem latex
\usepackage[polish]{babel}
\usepackage[utf8]{inputenc}

% dodatkowe pakiety
\usepackage{enumerate}
\usepackage{listings}
\lstloadlanguages{TeX}

\lstset{
  literate={ą}{{\k{a}}}1
           {ć}{{\'c}}1
           {ę}{{\k{e}}}1
           {ó}{{\'o}}1
           {ń}{{\'n}}1
           {ł}{{\l{}}}1
           {ś}{{\'s}}1
           {ź}{{\'z}}1
           {ż}{{\.z}}1
           {Ą}{{\k{A}}}1
           {Ć}{{\'C}}1
           {Ę}{{\k{E}}}1
           {Ó}{{\'O}}1
           {Ń}{{\'N}}1
           {Ł}{{\L{}}}1
           {Ś}{{\'S}}1
           {Ź}{{\'Z}}1
           {Ż}{{\.Z}}1
}

%---------------------------------------------------------------------------

\author{Tomasz Kasprzyk, Daniel Ogiela, Jakub Stępak}
\shortauthor{T. Kasprzyk, D. Ogiela, J.Stępak}

\titlePL{System obliczający wyniki wyborów dla uogólnienia systemu k-Borda}

\shorttitlePL{System obliczający wyniki wyborów dla uogólnienia systemu k-Borda} % skrócona wersja tytułu jeśli jest bardzo długi

\thesistypePL{Dokumentacja procesowa}

\supervisorPL{dr hab. inż. Piotr Faliszewski}

\date{2016}

\departmentPL{Katedra Informatyki}

\facultyPL{Wydział Informatyki, Elektroniki i Telekomunikacji}

\setlength{\cftsecnumwidth}{10mm}

%---------------------------------------------------------------------------

\begin{document}

\titlepages

\tableofcontents
\clearpage

%\chapter{Wprowadzenie}
\label{cha:wprowadzenie}

\LaTeX~jest systemem składu umożliwiającym tworzenie dowolnego typu dokumentów (w~szczególności naukowych i technicznych) o wysokiej jakości typograficznej (\cite{Dil00}, \cite{Lam92}). Wysoka jakość składu jest niezależna od rozmiaru dokumentu -- zaczynając od krótkich listów do bardzo grubych książek. \LaTeX~automatyzuje wiele prac związanych ze składaniem dokumentów np.: referencje, cytowania, generowanie spisów (treśli, rysunków, symboli itp.) itd.

\LaTeX~jest zestawem instrukcji umożliwiających autorom skład i wydruk ich prac na najwyższym poziomie typograficznym. Do formatowania dokumentu \LaTeX~stosuje \TeX a (wymiawamy 'tech' -- greckie litery $\tau$, $\epsilon$, $\chi$). Korzystając z~systemu składu \LaTeX~mamy za zadanie przygotować jedynie tekst źródłowy, cały ciężar składania, formatowania dokumentu przejmuje na siebie system.

%---------------------------------------------------------------------------

\section{Cele pracy}
\label{sec:celePracy}

Celem poniższej pracy jest zapoznanie studentów z systemem \LaTeX~w zakresie umożliwiającym im samodzielne, profesjonalne złożenie pracy dyplomowej w systemie \LaTeX.


%---------------------------------------------------------------------------

\section{Zawartość pracy}
\label{sec:zawartoscPracy}

W rodziale~\ref{cha:pierwszyDokument} przedstawiono podstawowe informacje dotyczące struktury dokumentów w \LaTeX u. Alvis~\cite{Alvis2011} jest językiem 



















%\chapter{Pierwszy dokument}
\label{cha:pierwszyDokument}

W rozdziale tym przedstawiono podstawowe informacje dotyczące struktury prostych plików \LaTeX a. Omówiono również metody kompilacji plików z zastosowaniem programów \emph{latex} oraz \emph{pdflatex}.

%---------------------------------------------------------------------------

\section{Struktura dokumentu}
\label{sec:strukturaDokumentu}

Plik \LaTeX owy jest plikiem tekstowym, który oprócz tekstu zawiera polecenia formatujące ten tekst (analogicznie do języka HTML). Plik składa się z dwóch części:
\begin{enumerate}%[1)]
\item Preambuły -- określającej klasę dokumentu oraz zawierającej m.in. polecenia dołączającej dodatkowe pakiety;

\item Części głównej -- zawierającej zasadniczą treść dokumentu.
\end{enumerate}


\begin{lstlisting}
\documentclass[a4paper,12pt]{article}      % preambuła
\usepackage[polish]{babel}
\usepackage[utf8]{inputenc}
\usepackage[T1]{fontenc}
\usepackage{times}

\begin{document}                           % część główna

\section{Sztuczne życie}

% treść
% ąśężźćńłóĘŚĄŻŹĆŃÓŁ

\end{document}
\end{lstlisting}

Nie ma żadnych przeciwskazań do tworzenia dokumentów w~\LaTeX u w~języku polskim. Plik źródłowy jest zwykłym plikiem tekstowym i~do jego przygotowania można użyć dowolnego edytora tekstów, a~polskie znaki wprowadzać używając prawego klawisza \texttt{Alt}. Jeżeli po kompilacji dokumentu polskie znaki nie są wyświetlane poprawnie, to na 95\% źle określono sposób kodowania znaków (należy zmienić opcje wykorzystywanych pakietów).


%---------------------------------------------------------------------------

\section{Kompilacja}
\label{sec:kompilacja}


Załóżmy, że przygotowany przez nas dokument zapisany jest w pliku \texttt{test.tex}. Kolejno wykonane poniższe polecenia (pod warunkiem, że w pierwszym przypadku nie wykryto błędów i kompilacja zakończyła się sukcesem) pozwalają uzyskać nasz dokument w formacie pdf:
\begin{lstlisting}
latex test.tex
dvips test.dvi -o test.ps
ps2pdf test.ps
\end{lstlisting}
%
lub za pomocą PDF\LaTeX:
\begin{lstlisting}
pdflatex test.tex
\end{lstlisting}

Przy pierwszej kompilacji po zmiane tekstu, dodaniu nowych etykiet itp., \LaTeX~tworzy sobie spis rozdziałów, obrazków, tabel itp., a dopiero przy następnej kompilacji korzysta z tych informacji.

W pierwszym przypadku rysunki powinny być przygotowane w~formacie eps, a~w~drugim w~formacie pdf. Ponadto, jeżeli używamy polecenia \texttt{pdflatex test.tex} można wstawiać grafikę bitową (np. w formacie jpg).



%---------------------------------------------------------------------------

\section{Narzędzia}
\label{sec:narzedzia}


Do przygotowania pliku źródłowego może zostać wykorzystany dowolny edytor tekstowy. Niektóre edytory, np. Emacs, mają wbudowane moduły ułatwiające składanie tekstów w LaTeXu (kolorowanie składni, skrypty kompilacji, itp.).

Jednym z bardziej znanych środowisk do składania dokumentów  \LaTeX a jest {\em Kile}. Aplikacja dostępna jest dla środowiska KDE począwszy od wersji 2. Zawiera edytor z podświetlaną składnią, zestawy poleceń \LaTeX a, zestawy symboli matematycznych, kreatory tabel, macierzy, skrypty kompilujące i konwertujące podpięte są do poleceń w menu aplikacji (i pasków narzędziowych), dostępne jest sprawdzanie pisowni, edytor obsługuje projekty (tzn. dokumenty składające się z~wielu plików), umożliwia przygotowanie i~zarządzanie bibliografią, itp.

Na stronie \underline{\texttt{http://kile.sourceforge.net/screenshots.php}} zamieszczono kilkanaście zrzutów ekranu środowiska {\em Kile}, które warto przejrzeć, by wstępnie zapoznać się z~możliwościami programu.

Bardzo dobrym środowiskiem jest również edytor gEdit z wtyczką obsługującą \LaTeX a. Jest to standardowy edytor środowiska Gnome. Po instalacji wtyczki obsługującej \LaTeX a, edytor nie ustępuje funkcjonalnościom środowisku Kile, a jest zdecydowanie szybszy w działaniu. Lista dostępnych wtyczek dla tego edytora znajduje się pod adresem \underline{\texttt{http://live.gnome.org/Gedit/Plugins}}. Inne polecane wtyczki to: 
\begin{itemize}
\item Edit shortcuts -- definiowanie własnych klawiszy skrótu;
\item Line Tools -- dodatkowe operacje na liniach tekstu;
\item Multi-edit -- możliwość jednoczesnej edycji w wielu miejscach tekstu;
\item Zoom -- zmiana wielkości czcionki edytora z użyciem rolki myszy;
\item Split View -- możliwość podziału okna edytora na 2 części. 
\end{itemize}



%---------------------------------------------------------------------------

\section{Przygotowanie dokumentu}
\label{sec:przygotowanieDokumentu}

Plik źródłowy \LaTeX a jest zwykłym plikiem tekstowym. Przygotowując plik
źródłowy warto wiedzieć o kilku szczegółach:

\begin{itemize}
\item
Poszczególne słowa oddzielamy spacjami, przy czym ilość spacji nie ma znaczenia.
Po kompilacji wielokrotne spacje i tak będą wyglądały jak pojedyncza spacja.
Aby uzyskać {\em twardą spację}, zamiast znaku spacji należy użyć znaku {\em
tyldy}.

\item
Znakiem końca akapitu jest pusta linia (ilość pusty linii nie ma znaczenia), a
nie znaki przejścia do nowej linii.

\item
\LaTeX~sam formatuje tekst. \textbf{Nie starajmy się go poprawiać}, chyba, że
naprawdę wiemy co robimy.
\end{itemize} 



%----------------------------------------------------------------------------
\chapter{Cele projektu}
\label{cha:cele_projektu}
Celem projektu jest stworzenie systemu obliczającego wyniki wyborów dla uogólnienia opisanego poniżej systemu k-Borda. 
Stworzona aplikacja webowa ma pozwalać użytkownikowi na szybkie definiowanie wyborów, jak również importowanie istniejących danych, w celu uzyskania wyników. 
Jako wybory rozumiemy nie tylko te, w których głosujący wyłaniają swoich przedstawicieli w organach władzy - swoje preferencje można także określić dla, na przykład, filmów jakie chcemy obejrzeć podczas seansu z przyjaciółmi, w jaką grę chcielibyśmy zagrać, jaką restaurację wybrać na rodzinne spotkanie itp.
Ustalenie preferencji to jedno, ale o tym, kto lub co zostanie wybrane na podstawie preferencji wszystkich głosujących, decyduje również system wyborczy.
Określając odpowiednie parametry aplikacja ma za zadanie ilustrować jak, w zależności od parametrów, zmieniają się wyniki wyborów, przy tym samym zestawie preferencji.
W pierwszej kolejności użytkownik określa listę preferencji wyborców. Określając swoje preferencje każdy wyborca porządkuje kandydatów w kolejności od najlepszego do najgorszego. Lista preferencji może zostać wygenerowana losowo, zaimportowana z odpowiednio sformatowanego pliku lub stworzona ręcznie.
Na wprowadzonych do systemu listach preferencji i ustalonej liczbie k wyboraców użytkownik może wielokrotnie uruchamiać algorytm wyłaniania zwycięzców. Przed każdym uruchomieniem algorytmu ustalany jest parametr p normy $\ell_p$, który w głównej mierze (pomijając listę preferencji wszystkich głosujących) decyduje o wyniku obliczeń.


\chapter{Studium wykonalności}
\label{cha:studium_wykonalnosci}

\section{Opis stanu istniejącego}
\label{sec:opis_stanu_istniejacego}

\section{Wybór frameworku dla aplikacji webowej}
\label{sec:wybor_frameworku_dla_aplikacji_webowej}

% itd.
% \appendix
% \include{dodatekA}
% \include{dodatekB}
% itd.

\bibliographystyle{alpha}
\bibliography{bibliografia}
%\begin{thebibliography}{1}
%
%\bibitem{Dil00}
%A.~Diller.
%\newblock {\em LaTeX wiersz po wierszu}.
%\newblock Wydawnictwo Helion, Gliwice, 2000.
%
%\bibitem{Lam92}
%L.~Lamport.
%\newblock {\em LaTeX system przygotowywania dokumentów}.
%\newblock Wydawnictwo Ariel, Krakow, 1992.
%
%\bibitem{Alvis2011}
%M.~Szpyrka.
%\newblock {\em {On Line Alvis Manual}}.
%\newblock AGH University of Science and Technology, 2011.cccccc
%\newblock \\\texttt{http://fm.ia.agh.edu.pl/alvis:manual}.
%
%\end{thebibliography}

\end{document}
