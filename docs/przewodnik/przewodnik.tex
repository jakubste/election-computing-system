% By zmienic jezyk na angielski/polski, dodaj opcje do klasy english lub polish
\documentclass[polish,11pt]{aghthesis}
\usepackage[utf8]{inputenc}
\usepackage{url}
\usepackage{amsfonts}
\usepackage{caption}

\author{Tomasz Kasprzyk, Daniel Ogiela\\ Jakub Stępak}

\title{System obliczający wyniki wyborów dla uogólnienia systemu k-Borda}

\supervisor{dr hab.\ inż.\ Piotr Faliszewski}

\date{2016}

% Szablon przystosowany jest do druku dwustronnego, 
\begin{document}

\maketitle
\tableofcontents
\clearpage

%--------------------------Rozdział Wizja produktu -----------------------------------------
%-------------------------------------------------------------------------------------------

\section{Wizja produktu}

%--------------------------Podrozdział Opis problemu ---------------------------------------
\subsection{Opis problemu}
Projekt dotyczy obliczania wyników wyborów. Wybory i sposób wyłaniania zwycięzców są
ściśle określone. Wybory można opisać jako parę $(C, V)$, gdzie $C = \left\{c_1, c_2, ... , c_m\right\}$ stanowi zbiór kandydatów, a $V = \left\{v_1, v_2, ... , v_n\right\}$ to ciąg wyborców. Każdy z wyborców posiada swoje preferencje wyborcze, które są opisane przez ciąg kandydatów ze zbioru $C$. Długość ciągu kandydatów wynosi $m$, a kandydaci są uporządkowani od najbardziej preferowanego do najmniej preferowanego. Ponadto określony jest rozmiar $k$ zwycięskiego komitetu, który określa liczbę kandydatów, którzy zwyciężyli w wyborach.
%\vspace{5mm}

\begin{center}
\includegraphics[width=0.8\textwidth]{pics/przykladowe_wybory.png}
\end{center}
Ciąg kandydatów przyporządkowany danemu wyborcy traktowany jest jako głos w
wyborach. Każdemu kandydatowi w ciągu stanowiącym głos, przyporządkowane są punkty
według punktacji Bordy. Jeżeli przez $i$ zostanie oznaczona pozycja danego kandydata w
ciągu stanowiącym głos (jeżeli kandydat jest pierwszy w ciągu, wtedy $i = 1$, jeśli drugi, wtedy $i = 2$ itd.), to wartość punktowa przyporządkowana według metody Bordy w tym głosie dla tego kandydata wynosi $\beta(i) = m - i$ , gdzie $m$ to rozmiar zbioru $C$ (liczba wszystkich kandydatów). Zatem dla danego głosu kolejni uporządkowani kandydaci otrzymują kolejno $m - 1, m - 2, … , 1, 0$ punktów.

\begin{center}
\includegraphics[width=0.8\textwidth]{pics/Borda_points.png}
\end{center}
W celu wyłonienia zwycięzców wyborów komitetów definiuje się tzw. funkcję satysfakcji,
która bazuje między innymi na punktacji metodą Bordy. Funkcja satysfakcji określa
zadowolenie danego wyborcy ze zwycięskiego komitetu. W celu zwięzłego zdefiniowania
funkcji celu wprowadza się pojęcie ciągu pozycji, które określa dla danego wyborcy pozycje
wszystkich zwycięskich kandydatów z jego preferencji. Ciąg pozycji jest posortowany
rosnąco - pozycja najbardziej preferowanego kandydata spośród zwycięzców na początku
ciągu, pozycja najmniej preferowanego kandydata spośród zwycięzców na końcu ciągu.
\clearpage
\noindent
\emph{Przykład} \\ 
Dla następujących preferencji:
\begin{center}
\includegraphics[width=0.8\textwidth]{pics/positions.png}
\end{center}
komitetu $K = \left\{a, b, e\right\}$ i wyborcy $v_3$ ciąg pozycji oznaczany ${pos_v}_3$ wynosi $(3,4,6)$, gdyż pozycja najlepszego kandydata ze zbioru $K$ wynosi $3$ (kandydat $e$), druga najlepsza to pozycja nr $4$ (kandydat $b$), a ostatnią pozycją najmniej preferowanego kandydata jest pozycja nr $6$ (kandydat $a$). \\ \\
W tym momencie można zdefiniować funkcję satysfakcji, która jako argument przyjmuje
zdefiniowany przed chwilą ciąg pozycji, który dla każdego wyborcy $i$ dla każdego komitetu
jest określony jednoznacznie. Funkcja satysfakcji określająca zadowolenie danego wyborcy
ze zwycięskiego komitetu, będzie więc oznaczana jako $f(i_1, i_2, ... , i_k)$. \\ \\
Poprzez różne zdefiniowanie funkcji satysfakcji, definiowane są odmienne systemy
wyborcze. Znanymi systemami wyborczymi są system $k-Borda$ oraz system
$Chamberlin-Courant’a$. Funkcje satysfakcji dla tych systemów określone są odpowiednio:
$$f_{k-Borda}(i_1, i_2, ..., i_k) = \beta(i_1) + \beta(i_2) + ... + \beta(i_k) 
$$ $$f_{CC}(i_1, i_2, ..., i_k) = \beta(i_1) = max(\beta(i_1), \beta(i_2), ..., \beta(i_k))$$
Wyniki wyborów dla podanych systemów mogą być różne (i zazwyczaj są) pomimo takich
samych danych wejściowych. Inaczej rzecz ujmując, zwycięskie komitety mogą składać się z
różnych kandydatów po zastosowaniu odmiennych funkcji satysfakcji. \\ \\
Funkcja satysfakcji, która dotyczy niniejszego projektu inżynierskiego jest oparta na normie
$\ell_p$ . Norma $\ell_p$ zdefiniowana jest następująco:
$$\ell_p(x_1,x_2,\dots, x_n) = \sqrt[p]{x_1^p+x_2^p+\dots+x_n^p}$$
, gdzie $x_1, x_2, ..., x_n \in \mathbb{R}, p \in \mathbb{N}$ \\ \\
Dla skrajnych przypadków, gdy $p = 1$ i gdy $p \to \infty$ norma $\ell_p$ przyjmuje odpowiednio postać sumy: $$l_1(x_1, x_2,\dots, x_n) = x_1 + x_2 +\dots+ x_n$$ oraz postać operatora maksimum: $$l_\infty(x_1, x_2,\dots, x_n) = max(x_1, x_2,\dots, x_n)$$
\\ Mając zdefiniowaną normę $\ell_p$ można określić funkcję satysfakcji będącą tematem pracy. $$f_{{\ell_p}-Borda}(i_1, i_2, ..., i_k) = \ell_p(\beta(i_1), \beta(i_2), ..., \beta(i_k)) = \sqrt[p]{[\beta(i_1)]^p+[\beta(i_2)]^p+\dots+[\beta(i_k)]^p}$$
\clearpage
Funkcja jest uogólnieniem systemu $k-Borda$. Jeżeli za $p$ przyjęta zostanie wartość $1$,
zdefiniowana funkcja przyjmuje postać funkcji satysfakcji dla zwykłego systemu $k-Borda$. Dla $p$ dążącego do nieskończoności funkcja przyjmuje postać funkcji satysfakcji dla systemu
$Chamberlin-Courant’a$.

% ------------------------------------------------------------------------------------------

%--------------------------Podrozdział Możliwości ------------------------------------------
\subsection{Możliwości}
Obliczanie wyników wyborów dla systemu $\ell_p-Borda$ metodą typu brute-force jest
czasochłonne. Już dla stosunkowo niewielkich rozmiarem danych, algorytm jest nieużyteczny. Dlatego głównym zadaniem projektu jest zaprojektowanie i implementacja algorytmów heurystycznych, które możliwie dokładnie i możliwie szybko obliczają wyniki wyborów. Ponadto do zadań zespołu należy próba oceny działania stworzonych algorytmów heurystycznych.

% ------------------------------------------------------------------------------------------

%--------------------------Motywacja do stworzenia projektu --------------------------------
\subsection{Motywacja do stworzenia projektu}
Jednym z głównym celów realizacji projektu jest chęć poznania różnych sposobów tworzenia
algorytmów heurystycznych oraz zastosowanie ich do rozwiązania problemów w
interesującej zespół tematyce, jaką jest sposób wyłaniania zwycięzców w wyborach i
związek między tym sposobem a satysfakcją wyborców ze zwycięzców. Ponadto zespół
chciał porównać pod różnymi względami różne algorytmy heurystyczne. Interesujące jest
zestawienie algorytmów pod względem dokładności obliczanych rozwiązań, czasu działania,
czy trudności w projektowaniu i implementacji. 

Drugą, istotną motywacją do stworzenia systemu jest możliwość wykorzystania go do wielu
bardzo różnych sytuacji. Zdefiniowanie wyborów, których wyniki ma liczyć system, może
dotyczyć różnego rodzaju wyborów. Mogą to być wybory ludzi do wszelakiego typu organów
władz na różnych szczeblach organizacji państwowych lub organizacji prywatnych. Wybory
nie muszą dotyczyć ludzi, a mogą polegać na selekcji innych rzeczy. Można zdefiniować
wybory na filmy, które zostaną odtworzone w trakcie podróży samolotem, czy podczas
seansu z rodziną i znajomymi. Mogą to być wybory na miejsca wspólnego wypadu znajomych na weekend. Wszędzie, gdzie ma sens zastosowanie preferencji (ułożenie
rzeczy od najbardziej do najmniej pożądanej) można wykorzystać stworzony system.

Dzięki stworzonemu systemowi według określonych wymagań, możliwe jest nie tylko
szybkie obliczenie wyborów na bazie wprowadzonych do systemu preferencji. Możliwe jest
sterowaniem sposobem obliczania wyborów według pożądanej reprezentatywności
zwycięskich kandydatów. Można decydować czy zwycięzcy wyborów będą tacy, że każdy
znajdzie wśród nich takiego, którego bardzo lubi, ale też taki, którego bardzo nie lubi. Czy
może zwycięzcy mają być tacy, że większość wyborców co prawda nie znajdzie wśród nich
bardzo pożądanego kandydata, ale jednocześnie nie znajdzie takiego, którego bardzo nie
lubi. Możliwy jest również wybór pośredni między wskazanymi dwoma rozwiązaniami.

Przykładowo, jeżeli użytkownik chce wybrać kilka filmów do wspólnego seansu z
przyjaciółmi i rodziną, to może zdecydować między różnymi opcjami. Jedną z opcji jest taki
wybór, aby wśród wybranych filmów każdy ze znajomych znalazł film, który mu bardzo
odpowiada, lecz jednocześnie znalazły się takie, które większości znajomych wyraźnie nie
odpowiadają. Druga możliwość to taka, w której wybrane zostaną filmy, które większości
znajomych ani za bardzo nie odpowiadają, ani za bardzo nie przeszkadzają. Pozostałymi
możliwościami są opcje pośrednie między wskazanymi.
\clearpage

% ------------------------------------------------------------------------------------------

%--------------------------Podrozdział Użytkownicy -----------------------------------------
\subsection{Użytkownicy}
Potencjalnymi użytkownikami systemu mogą być osoby i organizacje, które potrzebują
systemu wybierania kilku elementów spośród puli wszystkich elementów w taki sposób, aby
usatysfakcjonować swoich klientów. System może mieć przykładowe zastosowanie dla
przedsiębiorstw realizujących usługi transportu lotniczego czy autokarowego do umilenia
czasu podróży poprzez wybór odpowiednich filmów czy potraw. Ponadto system może
służyć pojedynczym osobom w celu wyboru odpowiedniej formy spędzania wolnego czasu w
grupie znajomych.

% ------------------------------------------------------------------------------------------

%--------------------------Podrozdział Usługi dostarczane przez system ---------------------
\subsection{Usługi dostarczanie przez system}

\subsubsection{Usługi wymagane}
\begin{itemize}
    \item możliwość wprowadzenia wyborów do systemu
    \item obliczenie wyników wyborów
    \item wyświetlanie wyników wyborów
\end{itemize}

\subsubsection{Usługi pożądane}
\begin{itemize}
    \item możliwość zarządzania swoimi wyborami (dodawanie nowych wyborów, nowych wyników, porównywanie różnych rezultatów)
    \item intuicyjna nawigacja po systemie
\end{itemize}

\subsubsection{Usługi dodatkowe}
\begin{itemize}
    \item wygodny sposób definiowania wyborów - zadanie projektowe koncentruje się na
projektowaniu i tworzeniu algorytmów obliczających wyniki wyborów. System musi
wczytywać pewien format pliku wejściowego definiującego wybory, który niekoniecznie jest prosty i intuicyjny w tworzeniu. Jeżeli ta dodatkowa usługa zostałaby zrealizowana, ułatwiłaby użytkowanie systemu i poprawiła zadowolenie użytkowników z korzystania z systemu
\end{itemize}

% ------------------------------------------------------------------------------------------

%-------------------------------------------------------------------------------------------
%-------------------------------------------------------------------------------------------
\section{Studium wykonalności}
  
\subsection{Opis wymagań}

\subsubsection{Wymagania funkcjonalne}
Poniżej zamieszczona jest lista najważniejszych wymagań w postaci kolejnych tabel. Tabela składa się z trzech pól opisujących dane wymaganie: \textit{Nazwa, Opis} i \textit{Uzasadnienie}. Pole \textit{Nazwa} jest nazwą usługi dostarczanej przez system, pole \textit{Opis} szczegółowo opisuje w jaki sposób jest realizowane dane wymaganie, a pole \textit{Uzasadnienie} podaje powód dlaczego w taki, a nie inny sposób wymaganie zostało zrealizowane i postawione.
\newpage
\begin{table}
%\centering
\begin{tabular}{|c|p{12.5cm}|}
\hline
\textbf{Nazwa} & Zdefiniowanie wyborów \\ 
\hline 
\textbf{Opis} & System oferuje różne sposoby zdefiniowania i wprowadzenia wyborów
do systemu. Pierwszym i podstawowym jest możliwość określania
wyborów w pliku formatu $.soc$. Sposób określania w nim kandydatów
oraz głosów wyborców jest ściśle określony. W pierwszej linii pliku
znajduje się liczba kandydatów do zwycięskiego komitetu. W kolejnych
liniach znajdują się numery i nazwy kandydatów. Następnie jest linia z
informacją o liczbie głosujących, liczbie policzonych głosów oraz liczbie
głosów unikalnych. W ostatnim bloku pliku znajdują się linie z
informacją o preferencjach głosów unikalnych. Każda linia składa się z
liczby powtórzeń danego głosu oraz preferencji określonej dla tego
głosu, która ma postać ciągu numerów kandydatów od najbardziej do
najmniej pożądanego. Po stworzeniu opisanego pliku system umożliwia
wskazanie pliku z systemu plików.

Drugim sposobem definiowania wyborów oferowanym przez system
jest generowanie wyborów z rozkładu normalnego. W tym celu
kandydaci i wyborcy są reprezentowani jako punkty płaszczyźnie.
Preferencje wyborców obliczane są na bazie odległości euklidesowych
do poszczególnych kandydatów na płaszczyźnie. Aby wygenerować
wybory użytkownik podaje w formularzu parametry potrzebne do
generacji wyborów zgodnie z rozkładem normalnym. Uzupełnia pola
dotyczące: liczby kandydatów, średniej wartości współrzędnej $x$ dla
wyborców, średniej wartości współrzędnej x dla kandydatów, średniej
wartości współrzędnej $y$ dla wyborców, średniej wartości współrzędnej
$y$ dla kandydatów, odchylenia standardowego wyborców oraz
odchylenia standardowego kandydatów. \\ 
\hline 
\textbf{Uzasadnienie} & Definicja wyborów za pomocą pliku formatu $.soc$ zapewnia spełnienie
podstawowej usługi oferowanej przez system - wprowadzenia wyborów
do systemu. Uzasadnieniem dla konieczności generacji wyborów z
rozkładu normalnego jest możliwość atrakcyjnej i klarownej wizualizacji
wyborów i wyników wyborów. Wizualizacja w ten sposób wyników
wyborów umożliwia między innymi ocenienie jakości stworzonych
algorytmów i porównanie ich.\\ 
\hline 
\end{tabular}
\caption{Zdefiniowanie wyborów} 
\end{table}

\begin{table}
%\centering
\begin{tabular}{|c|p{10cm}|}
\hline
\textbf{Nazwa} & Obliczanie wyników wyborów \\ 
\hline 
\textbf{Opis} & System oferuje kilka algorytmów heurystycznych obliczających wyniki
wyborów. Algorytmy bazują na różnym paradygmacie tworzenia ich.
Jeden z paradygmatów to programowanie zachłanne, a drugi to
programowanie genetyczne. Algorytmy dają przybliżone rozwiązania
problemu pod względem jego dokładności. Użytkownik przy tworzeniu
nowego wyniku dla zdefiniowanych wyborów może w prosty sposób
wybrać jeden z algorytmów. \\ 
\hline 
\textbf{Uzasadnienie} & Kilka stworzonych algorytmów i różne podejście do tworzenia ich,
umożliwia ocenę jakości algorytmów na bazie zestawienia wyników
obliczonych wyborów. Ponadto stworzenie różnych algorytmów pozwoli
na dogłębniejsze poznanie sposobów tworzenia algorytmów
heurystycznych, jak również samej tematyki pracy, jaką były różne
sposoby wyboru zwycięskiego komitetu.\\ 
\hline 
\end{tabular}
\caption{Obliczanie wyników wyborów} 
\end{table}

\begin{table}
%\centering
\begin{tabular}{|c|p{10cm}|}
\hline
\textbf{Nazwa} & Wersja algorytmu zachłannego $Greedy - CC$ \\ 
\hline 
\textbf{Opis} & Implementacja algorytmu zachłannego dla szczególnego przypadku
(parametr $p \to \infty$) \\ 
\hline 
\textbf{Uzasadnienie} & Dla stosunkowo "dużych" wartości parametru $p$ wykonywanie operacji
pierwiastkowania i potęgowania stanowi niepotrzebny narzut czasowy.
W tym przypadku operację obliczania normy $\ell_p$ można zastąpić
wyznaczaniem maksimum dla pojedynczych wyników kandydatów z
komitetu, dla którego liczono by normę $\ell_p$.
Korzyści jakie płyną z takiej modyfikacji to nie tylko skrócenie czasu
obliczeń, ale także ustalanie od jakiej wartości parametru $p$ komitet
wygrywający jest zgodny z wynikiem otrzymanym dla systemu
$Chamberlin-Courant’a$.\\ 
\hline 
\end{tabular} 
\caption{Wersja algorytmu zachłannego $Greedy - CC$}
\end{table}

\begin{table}
%\centering
\begin{tabular}{|c|p{10cm}|}
\hline
\textbf{Nazwa} & Prezentacja wyników wyborów \\ 
\hline 
\textbf{Opis} & System oferuje dwa sposoby prezentacji wyników wyborów. Pierwszy z
nich, to proste wskazanie kandydatów i wypisanie nazw kandydatów w
widocznym panelu. Ta forma prezentacji dotyczy wyników wyborów,
które zdefiniowane zostały za pomocą pliku formatu $.soc$.

Druga forma prezentacji wyników wyborów dotyczy wyborów wygenerowanych z
rozkładu normalnego. Wizualizacja polega na narysowaniu wykresu
$2D$, na którym punkty reprezentujące kandydatów ze zwycięskiego
komitetu są wyraźnie wyróżnione na tle kandydatów, którzy nie dostali
się do zwycięskiego komitetu. Dla wyborów generowanych z rozkładu normalnego system również oferuje wypisanie nazw zwycięzców w dobrze widocznym panelu.\\ 
\hline 
\textbf{Uzasadnienie} & Prezentacja wyników wyborów w postaci wykresu $2D$ z wyraźnie
zaznaczonymi zwycięzcami i przegranymi, którzy są reprezentowani
jako punkty, pozwala na porównanie w prosty sposób wyników
wyborów dla różnych wartości parametru $p$. Taka forma wizualizacji
jest jednocześnie wizualnie atrakcyjna.
Prezentacja wyników dla wyborów zdefiniowanych z pliku formatu $.soc$
ogranicza się jedynie do wypisania zwycięzców, ponieważ kandydaci
nie są reprezentowani jako punkty i tym samym nie mają
współrzędnych w tej odmianie wyborów.\\ 
\hline 
\end{tabular} 
\caption{Prezentacja wyników wyborów}
\end{table}

\clearpage
\subsubsection{Wymagania niefunkcjonalne}
Wymagania niefunkcjonalne podzielono na wymagania produktowe i wymagania organizacyjne. Poniżej zamieszczono tabele analogiczne do tych z wymagań funkcjonalnych dla odpowiednich typów wymagań niefunkcjonalnych.

\begin{center}
\textbf{Wymagania produktowe}
\end{center}
{
\centering
\begin{tabular}{|c|p{10cm}|}
\hline
\textbf{Nazwa} & Obliczanie wyników wyborów w satysfakcjonującym czasie \\ 
\hline 
\textbf{Opis} & Ponieważ dla już stosunkowo małych danych metoda brute-force
rozwiązywania opisanego problemu nie daje rezultatów w
satysfakcjonującym czasie, więc to wymaganie niefunkcjonalne jest
kluczowe w tym projekcie. Zastosowanie algorytmów heurystycznych
wychodzi naprzeciw temu wymaganiu. Algorytmy powinny dawać
rozwiązania w zadowalającym czasie dla możliwie dużych danych. \\ 
\hline 
\textbf{Uzasadnienie} & Czas działania algorytmu powinien się mieścić w możliwym do
zaakceptowania przedziale czasu ze względu na konieczność jego
użyteczności.\\ 
\hline 
\end{tabular}
\captionof{table}{Obliczanie wyników wyborów w satysfakcjonującym czasie}
}

\begin{center}
\textbf{Wymagania organizacyjne}
\end{center}
{
\centering
\begin{tabular}{|c|p{10cm}|}
\hline
\textbf{Nazwa} & Dotrzymanie terminów na przedstawienie poszczególnych elementów
pracy \\ 
\hline 
\textbf{Opis} & Kolejne elementy pracy, które należało przedstawiać w umówionych
terminach to: wizja produktu, studium wykonalności, prototyp oraz tuż
przed końcem projektu: produkt końcowy wraz z dokumentacją. \\ 
\hline 
\textbf{Uzasadnienie} & Terminy są dostosowane do trybu wykonywania pracy inżynierskiej,
który jest ustalany przez władze uczelni.\\ 
\hline 
\end{tabular}
\captionof{table}{Obliczanie wyników wyborów w satysfakcjonującym czasie}
}

\subsection{Strategia testowania}
Podstawowym rodzajem testów pisanych w trakcie tworzenia systemu były testy
jednostkowe. Były one dodawane na bieżąco od razu przy dodawaniu kolejnych
funkcjonalności systemu. 

Ponadto w celu zagwarantowania poprawności działania wcześniej dodanych
funkcjonalności po dodaniu nowych skorzystano z ciągłej integracji. Do tego celu
wykorzystano serwis Travis Cl skonfigurowany do repozytorium kodu. Przy każdej zmianie
kodu źródłowego w repozytorium dokonywane było automatyczne włączenie testów
jednostkowych.  

Do oceny poprawności obliczanych wyników wyborów przez stworzone algorytmu,
przeprowadzono testy porównawcze. Zestawiano wyniki wyborów obliczonych przez różne
algorytmy dla tych samych danych wejściowych. Ponadto porównywano wykresy wyników
wyborów z oczekiwanymi rezultatami.
\clearpage

\subsection{Aspekt technologiczny}
Do stworzenia systemu wykorzystano język $Python 2.7$. Produkt postanowiono wykonać w
postaci aplikacji webowej. Aplikację internetową oparto na frameworku pythonowym $Django
1.9$. Do tworzenia stron internetowych użyto frameworku $Bootstrap$. Do wdrożenia systemu
wykorzystano platformę $Heroku$.

Wybrano taki stos technologiczny ze względu na doświadczenie części zespołu pracy z nim.
Wiedza na temat tych narzędzi przekonywała zespół o możliwości zrealizowania projektu w
tej technologii.

\subsection{Analiza ryzyka}
Głównym zagrożeniem dla realizacji produktu mogły być problemy ze stworzeniem
wystarczająco dobrego algorytmu heurystycznego dla opisanego problemu. Ze względu na
niewielkie doświadczenie zespołu w tej dziedzinie taka sytuacja mogła wystąpić. Głównym
zabezpieczeniem na wypadek takiego scenariusza było stosunkowo szybkie stworzenie
bazy całego systemu i możliwość wczesnego skupienia się na algorytmach heurystycznych.

Innym, dużo mniejszym zagrożeniem dla projektu mógł być brak skoordynowania prac
poszczególnych członków zespołu. Ponieważ dla członków zespołu projekt inżynierski był
tylko jednym z wielu zajęć w czasie trwania projektu, mogła wystąpić sytuacja kiedy ze
względu na natężenie innych obowiązków, dany członek zespołu nie mógłby wykonać
koniecznej pracy w odpowiednim czasie.

% o ile to mozliwe prosze uzywac odwolan \cite w konkretnych miejscach a nie \nocite

%\nocite{artykul2011,ksiazka2011,narzedzie2011,projekt2011}

\bibliography{bibliografia}

\end{document}
