\documentclass[pdflatex,11pt]{../aghdoc_version2}
% \documentclass{../aghdoc}               % przy kompilacji programem latex
\usepackage[polish]{babel}
\usepackage[utf8]{inputenc}

% dodatkowe pakiety
\usepackage{enumerate}
\usepackage{listings}
\lstloadlanguages{TeX}

\lstset{
  literate={ą}{{\k{a}}}1
           {ć}{{\'c}}1
           {ę}{{\k{e}}}1
           {ó}{{\'o}}1
           {ń}{{\'n}}1
           {ł}{{\l{}}}1
           {ś}{{\'s}}1
           {ź}{{\'z}}1
           {ż}{{\.z}}1
           {Ą}{{\k{A}}}1
           {Ć}{{\'C}}1
           {Ę}{{\k{E}}}1
           {Ó}{{\'O}}1
           {Ń}{{\'N}}1
           {Ł}{{\L{}}}1
           {Ś}{{\'S}}1
           {Ź}{{\'Z}}1
           {Ż}{{\.Z}}1
}

%---------------------------------------------------------------------------

\author{Tomasz Kasprzyk, Daniel Ogiela, Jakub Stępak}
\shortauthor{T. Kasprzyk, D. Ogiela, J.Stępak}

\titlePL{System obliczający wyniki wyborów dla uogólnienia systemu k-Borda}

\shorttitlePL{System obliczający wyniki wyborów dla uogólnienia systemu k-Borda} % skrócona wersja tytułu jeśli jest bardzo długi

\thesistypePL{Dokumentacja techniczna}

\supervisorPL{dr hab. inż. Piotr Faliszewski}

\date{2016}

\departmentPL{Katedra Informatyki}

\facultyPL{Wydział Informatyki, Elektroniki i Telekomunikacji}

\setlength{\cftsecnumwidth}{10mm}

% umożliwienie żeby domyślnie dokument nie robił wcięć poza wybranymi (\indent w tym miejscu)

\newlength\tindent
\setlength{\tindent}{\parindent}
\setlength{\parindent}{0pt}
\renewcommand{\indent}{\hspace*{\tindent}}

%---------------------------------------------------------------------------

\begin{document}

\titlepages

\tableofcontents


%----------------------------------------------------------------------------
\chapter{Dziedzina problemu}
\label{cha:dziedzina_problemu}

\section{Metoda obliczania wyników wyborów}
\label{sec:metoda_obliczania_wynikow_wyborow}

\subsection{Metoda Bordy}
\label{subsec:metoda_bordy}

Niech $v$ będzie głosem nad zbiorem kandydatów $C$. Wynik według Bordy kandydata $c$ w $v$ jest równy $\beta(i)=C-i$, gdzie $i$- pozycja kandydata w ciągu $v$.
Wynik $c$ w wyborach jest sumą wyników $c$ u każdego z wyborców


\subsection{Metoda k-Borda}
\label{subsec:metoda_k_borda}

Rozszerzenie metody Bordy. Wynik, zamiast dla jednego kandydata, obliczany jest dla ciągu kandydatów. $f_{kB}$- funkcja zadowolenia z komitetu. Ciąg $(i_1,\dots, i_k)$- ciąg pozycji kandydatów

\paragraph{Przyklad}
$C={c_1,c_2,c_3,c_4}$ - zbiór kandydatów, 
$v=(c_2,c_1,c_4,c_3)$ - głos
Niech $k = 2$ (wybory $2$ spośród $4$)
$w=(c_4,c_3)$
Najpierw określamy pozycje kandydatów z komitetu $w$ w $v$:
$pos_v(w)=(3,4)$, zatem wynik komitetu w dla głosu v wynosi
$f_{kB}(3,4) = (3) + (4) = ||C|| - 3 ) + ( ||C|| - 4 ) = 1 + 0 = 1$


\subsection{Uogólnienie - system $\ell_p$ Borda}
\label{subsec:system_ell_p_borda}

Zanim wprowadzone zostanie pojęcie uogólnionego systemu k-Borda warto przypomnieć wzór na normę $\ell_p$

\subsubsection{Norma $\ell_p$}
\label{subsubsection:norma_ell_p}

$\ell_p(x_1,x_2,\dots, x_n) = \sqrt[p]{x_1^p+x_2^p+\dots+x_n^p},$ 


Wówczas, w uogólnionej wersji metody k-Borda, funkcja zadowolenia $f_{kB}$ zostaje uzależniona również od parametru $p$ z powyższego wzoru. Norma liczona jest z wyników według Bordy, $\beta(i)$. Wzór uogólniony funkcji zadowolenia przyjmuje zatem postać:\\
$f_{\ell_pB}(p,(i_11,\dots,i_k))=p[(i1)]p+[(i2)]p+...+[(ik)]p$


Systemy  k-Borda i Cahmberlin’a-Courant’a są szczególnymi przypadkami zdefiniowanego powyżej systemu $\ell_p$- Borda:\\
Dla $p = 1$, $l_1$\\
$f_{\ell_pB}(1,(i_1, \dots ,i_k)) = \beta(i_1) + \beta(i_2) + \dots + \beta(ik) = f_{kB}(i_11, \dots,i_k)$\\
Dla $p= \infty$,  $l_{\infty}={max}$\\
$f_{\ell_pB}(\infty,(i_1, \dots ,i_k))=\lim\limits_{p \to \infty} \sqrt[p]{\beta\left[(i_1)\right]^p+\beta\left[(i_2)\right]^p+ \dots +\left[\beta(i_k)\right]^p}=\max{\beta(i_1),\beta(i_2),\dots,\beta(i_k)} = \beta(i_1)=f_{CC}$

\section{Format danych wejściowych}
\label{sec:format_danych_wejsciowych}
Pojedynczy plik składa się z następującego formatu:\\
\texttt{<liczba kandydatów> \\
1, <nazwa kandydata> \\
2, <nazwa kandydata> \\
… \\
<liczba kandydatów>, <nazwa kandydata> \\
<liczba głosujących>, <liczba głosów policzonych>, <liczba unikalnych głosów> \\
<liczba powtórzeń głosu>, <głos> \\
… \\
<liczba powtórzeń głosu>, <głos>}

\chapter{Architektura Django}
$Django$ to framework webowy napisany w $Pythonie$. Dostarcza wysokopoziomowych
abstrakcji pozwalających na szybkie i wygodne pisanie przejrzystych aplikacji. \\ \\
Jego architektura koncepcyjnie przypomina wzorzec architektoniczny $Model-View-Controller$,
jednak jak przyznają sami twórcy [ref], $Django$ nie do końca wpasowuje się w klasyczne
ujęcie $MVC$. \\ \\
$Django$ dostarcza mapowania obiekto-relacyjnego, dzięki którym całość modeli można ująć
w $Pythonie$. Wygodny $ORM$ zazwyczaj wystarcza do obsługi bazy danych, jednak zawsze
istnieje możliwość użycia bezpośrednio $SQL$. \\ \\
Widoki w $Django$ spełniają dwojaką funkcję - służą zarówno przekazaniu danych do
wyświetlenia, jak i ich modyfikacji. W wyświetleniu danych użytkownikowi pośredniczą
szablony ($templates$), które "opakowują" przekazane dane do postaci $HTMLa$, który może
wyświetlić przeglądarka internetowa. Dzięki temu wybór danych, jakie mają zostać pokazane
użytkownikowi jest oddzielony od samego sposobu ich prezentacji. \\ \\
Za kontroler z klasycznego $MVC$ można uznać sam framework dostarczający wspomnianej
obsługi bazy danych czy mapowania adresów URL do poszczególnych widoków.

\chapter{Opis modułów}
\label{cha:opis_modulow}

\section{Moduł administracji kont}
\subsection{Opis ogólny}
Moduł odpowiedzialny za zarządzanie kontami użytkowników. Umożliwia czynności
logowania, wylogowania oraz rejestracji. Nie współpracuje z innymi modułami. Korzysta
jedynie z bazy danych w celu weryfikacji użytkowników.

\subsection{Komponenty programowe}
Komponenty programowe dla tego modułu znajdują się w pakiecie $ecs.accounts$ \\ \\
{
\centering
\begin{tabular}{|c|c|c|}
\hline 
\textbf{Typ komponentu} & \textbf{Komponenty} & \textbf{Wykorzystanie} \\ 
\hline 
widoki & klasa $LoginView$ & logowanie \\ 
\cline{2-3} 
 & klasa $RegisterView$ & rejestracja \\ 
\hline 
szablony & $login.html$ & logowanie \\ 
\cline{2-3} 
 & $register.html$ & rejestracja \\ 
\hline 
formularze & klasa $LoginForm$ & logowanie \\ 
\cline{2-3} 
 & klasa $RegistrationForm$ & rejestracja \\ 
\hline 
\end{tabular} 
}

\section{Moduł zarządzania wyborami}
\subsection{Opis ogólny}
Moduł odpowiedzialny za administrację wyborami i ich wynikami. Umożliwia podstawowe
operacje na wyborach i ich wynikach oraz nawigację między nimi. Pozwala na wyświetlenie
listy wszystkich wyborów i ich wyników, stworzenie nowych wyborów lub wyniku, czy
usunięcie wyborów. W celu wykonania niektórych zadań współpracuje z modułem obliczania
wyników wyborów oraz modułem wizualizacji wyborów i ich wyników. Jest to najbardziej
rozbudowany moduł.

\subsection{Komponenty programowe}
% \usepackage{array} is required
\begin{tabular}{|c|c|p{5cm}|}
\hline 
\textbf{Typ komponentu} & \textbf{Komponenty} & \textbf{Wykorzystanie} \\ 
\hline 
widoki & klasa $ElectionListView$ & wyświetlenie listy wszystkich wyborów \\ 
\cline{2-3} 
 & klasa $ElectionCreateView$ & stworzenie wyborów \\ 
\cline{2-3} 
 & klasa $ElectionDeleteView$ & usunięcie wyborów \\ 
\cline{2-3} 
 & klasa $ElectionDetailView$ & wyświetlenie informacji szczegółowych o
danych wyborach \\ 
\cline{2-3} 
 & klasa $ResultCreateView$ & stworzenie wyniku dla danych wyborów -
wyświetlenie formularza określającego
parametry wyniku \\ 
\cline{2-3} 
 & klasa $ResultDetailsView$ & wyświetlenie danego wyniku danych
wyborów \\ 
\cline{2-3} 
 & klasa $ResultDeleteView$ & usuwanie pojedynczego wyniku wyborów \\ 
\hline 
szablony & $election_list.html$ & wyświetlenie listy wszystkich wyborów,
usunięcie wyborów - strona sukcesu
wyświetlana po usunięciu wyborów \\ 
\cline{2-3} 
 & $election_create.html$ & stworzenie wyborów \\ 
\cline{2-3} 
 & $election_details.html$ & stworzenie wyborów - strona sukcesu
wyświetlana po stworzeniu wyborów,
wyświetlenie informacji szczegółowych o
danych wyborach \\ 
\cline{2-3} 
 & $election_delete.html$ & usunięcie wyborów \\ 
\cline{2-3} 
 & $result_create.html$ & stworzenie nowego wyniku wyborów -
strona z formularzem \\ 
\cline{2-3} 
 & $result_details.html$ & wyświetlenie danego wyniku danych
wyborów \\ 
\cline{2-3} 
 & $result_delete.html$ & potwierdzenie usunięcia rezultatu \\ 
\hline 
formularze & klasa $ElectionForm$ & stworzenie wyborów \\ 
\cline{2-3}
 & klasa $ResultForm$ & stworzenie wyniku dla danych wyborów -
formularz określający parametry wyniku \\
\hline 
\end{tabular} 

\section{Moduł wizualizacji wyborów i wyników wyborów}
\subsection{Opis ogólny}
Moduł odpowiedzialny za stworzenie wykresu $2D$ wizualizującego wybory i jego wyniki.
Wizualizacja dotyczy tylko wyborów wygenerowanych z rozkładu normalnego. Wyborcy i
kandydaci są reprezentowani jako punkty na płaszczyźnie. Punkty reprezentujące wyborców
i kandydatów mają na wykresie odmienne kolory. Na wykresie wyników wyborów punkty
reprezentujące zwycięzców wyborów są powiększone. Moduł współpracuje z modułem
zarządzania wyborów, który zleca mu zadanie wizualizacji wyborów lub jego wyników. W
celu wykonania zadania moduł wizualizacji wyborów i wyników wyborów pobiera dane z
bazy danych.

\subsection{Komponenty programowe}
\begin{tabular}{|c|c|p{5cm}|}
\hline 
\textbf{Typ komponentu} & \textbf{Komponenty} & \textbf{Wykorzystanie} \\ 
\hline 
widoki & klasa $ScatterChartMixin$ & pobranie danych potrzebnych do
wygenerowania wykresów (punkty, tytuł
wykresu, serii danych), dziedziczy po
klasie $View$ \\ 
\cline{2-3} 
 & klasa $ElectionChartView$ & wizualizacja wyborów, klasa dziedziczy
po klasie $ScatterChartMixin$, pobranie
współrzędnych kandydatów i wyborców \\ 
\cline{2-3} 
 & klasa $ResultChartView$ & wizualizacja wyników wyborów, klasa
dziedziczy po klasie $ScatterChartMixin$,
pobranie współrzędnych kandydatów,
wyborców oraz zwycięzców \\ 
\hline 
\end{tabular} 

\section{Moduł generacji i wczytywania wyborów}
\subsection{Opis ogólny}
Moduł odpowiedzialny za generację wyborów z rozkładu normalnego oraz wczytywanie
wyborów z pliku formatu $.soc$. Moduł współpracuje z modułem zarządzania wyborami,
któremu zleca po wykonaniu swoich zadań, stworzenie i wysłanie użytkownikowi
odpowiedniej strony internetowej. Moduł zapewnia wygenerowanie wyborów z rozkładu normalnego według wskazanych parametrów oraz walidację danych przy wczytywaniu
wyborów z pliku. Po stworzeniu wyborów moduł komunikuje się z bazą danych w celu
utrwalenia wyborów.

\subsection{Komponenty programowe}
\begin{tabular}{|c|c|p{5cm}|}
\hline 
\textbf{Typ komponentu} & \textbf{Komponenty} & \textbf{Wykorzystanie} \\ 
\hline 
Widoki & klasa $ElectionLoadDataFormView$ & wczytanie wyborów z pliku \\ 
\cline{2-3} 
 & klasa
$ElectionGenerateDataFormView$ & generacja wyborów z rozkładu
normalnego \\ 
\hline 
Szablony & $election_load_data.html$ & strona z formularzem do
wskazania pliku \\ 
\cline{2-3} 
 & $election_generate_data.html$ & strona z formularzem do
wskazania parametrów wyborów i
rozkładu normalnego \\ 
\cline{2-3} 
 & $election_details.html$ & strona wyświetlana po
poprawnym wczytaniu danych z
pliku \\ 
\hline 
Formularze & klasa $ElectionLoadDataForm$ & wczytanie z pliku \\ 
\hline 
\end{tabular} 

\section{Moduł obliczania wyników wyborów}
\subsection{Opis ogólny}
Moduł odpowiedzialny za obliczanie wyników wyborów. Zapewnia różne algorytmy do
wykonania zadania. Użytkownik ma wybór między algorytmem genetycznym, dwoma
algorytmami zachłannymi oraz algorytmem typu brute-force. Moduł współpracuje z modułem
zarządzania wyborami, który zleca mu wykonanie zadania.
\subsection{Komponenty programowe}
Wszystkie komponenty programowe dotyczące modułu obliczania wyników wyborów
zawierają się w pakiecie $ecs.elections.algorithms$. \\
Klasy odpowiedzialne za poszczególne algorytmy:
\begin{itemize}
\item $BruteForce$ - odpowiedzialna za algorytm typu brute-force
\item $GreedyAlgorithm$ - odpowiedzialna za algorytm zachłanny zależny od parametru $p$
\item $GreedyCC$ - odpowiedzialna za algorytm zachłanny niezależny od parametru $p$
\item $GeneticAlgorithm$ - odpowiedzialna za algorytm genetyczny
\end{itemize}

\section{Moduł URL Resolver}
\subsection{Opis ogólny}
Moduł odpowiedzialny za przekazywanie żądań otrzymywanych przez klienta do
odpowiednich modułów.

\subsection{Komponenty programowe}
Przyporządkowania żądań użytkownika do odpowiednich modułów (adresów URL do
widoków) znajdują się w pliku $ecs.elections.urls.py$.

\bibliographystyle{alpha}
\bibliography{bibliografia}

\end{document}
