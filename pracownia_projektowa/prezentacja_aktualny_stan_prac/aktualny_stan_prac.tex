\documentclass{beamer}

\usepackage[T1]{fontenc}
\usepackage[polish]{babel}
\usepackage[utf8]{inputenc}
\usepackage{lmodern}
\usepackage{amsmath}

\selectlanguage{polish}

\mode<presentation> {
	\usetheme{Copenhagen}
}

\usepackage{graphicx} % Allows including images
\usepackage{booktabs} % Allows the use of \toprule, \midrule and \bottomrule in tables

\newcommand{\red}[1]{
	{ \color{red}{#1} }
}

\newcommand{\score}[2]{
	\stackrel
	{\red{#1}}
	{#2}
}

\definecolor{links}{HTML}{2A1B81}
\hypersetup{colorlinks,linkcolor=,urlcolor=links}

%----------------------------------------------------------------------------------------
%	TITLE PAGE
%----------------------------------------------------------------------------------------

\title
[System obliczający wyniki wyborów]
{System obliczający wyniki wyborów dla uogólnienia systemu k-Borda - aktualny stan prac}

\author
[T. Kasprzyk, D. Ogiela, J. Stępak]
{Tomasz Kasprzyk, \ Daniel Ogiela, \ Jakub Stępak}

\institute
[AGH]
{
Akademia Górniczo-Hutnicza

Wydział Informatyki, Elektroniki i Telekomunikacji

Katedra Informatyki 

}
\date{26 października 2016}

\begin{document}

\frame{\titlepage}

\begin{frame}
\frametitle{Plan prezentacji}
\tableofcontents
\end{frame}

%----------------------------------------------------------------------------------------
%	PRESENTATION SLIDES
%----------------------------------------------------------------------------------------

% ----------------------------------------------------
\section{Zrealizowane zadania}
% ----------------------------------------------------

\subsection{Obsługiwany format wyborów}

\begin{frame}

\frametitle{Obsługiwany format wyborów}
\begin{itemize}
\item Wczytywanie wyborów z pliku w formacie .soc
\item Generacja wyborów z rozkładu normalnego
\end{itemize}

\end{frame}

% ----------------------------------------------------

\subsection{Interfejs i wdrożenie systemu}

\begin{frame}

\frametitle{Interfejs i wdrożenie systemu}
\begin{itemize}
\item interfejs webowy oparty na Django 1.9
\item automatyczny interfejs administracyjny
\item platforma Heroku
\end{itemize}

\end{frame}

% ----------------------------------------------------

\subsection{Struktura danych w systemie i ich przechowywanie}

\begin{frame}

\frametitle{Struktura danych w systemie i ich przechowywanie}
\begin{itemize}
\item obiektowa struktura danych
\item mapowanie obiektowo-relacyjne za pomocą wbudowanego interfejsu
\end{itemize}

\end{frame}

% ----------------------------------------------------

\subsection{Implementacja operacji związanych z wyborami}

\begin{frame}

\frametitle{Tworzenie i usuwanie wyborów}
\begin{itemize}
\item możliwość wskazania nazwy i liczności zwycięskiego komitetu
\item możliwość wskazania pliku w odpowiednim formacie
\item możliwość wskazania parametrów do wygenerowania danych z rozkładu normalnego
\item stworzenie wykresu
\item usuwanie całych wyborów 
\end{itemize}

\end{frame}

% ----------------------------------------------------

\begin{frame}

\frametitle{Dodawanie wyników wyborów}
\begin{itemize}
\item możliwość wybrania parametru p
\item możliwość wskazania algorytmu 
\end{itemize}

\end{frame}

% ----------------------------------------------------

\begin{frame}

\frametitle{Wyświetlanie wyników wyborów}
\begin{itemize}
\item stworzenie w głównym oknie wyborów listy wszystkich rezultatów
\item możliwość przejścia do szczegółowych wyników wyborów po kliknięciu na link
\item wyświetlenie wykresu z wynikami i listy zwycięskich nazwisk 
\end{itemize}

\end{frame}

% ----------------------------------------------------

\begin{frame}

\frametitle{Obliczanie wyników wyborów}
\begin{itemize}
\item algorytm brute-force
\item algorytm zachłanny
\item algorytm genetyczny
\end{itemize}

\end{frame}

% ----------------------------------------------------



% ----------------------------------------------------
\section{Zadania pozostałe do realizacji}
% ----------------------------------------------------

\subsection{Co zostało}

\begin{frame}

\frametitle{Co zostało}
\begin{itemize}
\item przyspieszenie działania algorytmu genetycznego
\item testy porównawcze
\item dokumentacja
\end{itemize}

\end{frame}

% ----------------------------------------------------

\subsection{Co ewentualnie}

\begin{frame}

\frametitle{Co ewentualnie}
\begin{itemize}
\item ulepszenie sposobu wyświetlania wyników wyborów
\item nowy sposób generowania wyborów
\item poprawa UX
\end{itemize}

\end{frame}

% ----------------------------------------------------

\begin{frame}
\Huge{\centerline{Dziękujemy za uwagę}}
\end{frame}

\end{document}
