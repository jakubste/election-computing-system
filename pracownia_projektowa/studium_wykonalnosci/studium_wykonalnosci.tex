%%%%%%%%%%%%%%%%%%%%%%%%%%%%%%%%%%%%%%%%%
% Election Computing System
% System obliczający i wizualizujący wyniki zgodnie z uogólnieniem systemu k-Borda na podstawie norm ell_p.
%
% Prezentacja wykonana na Pracownię Projektową
%
%%%%%%%%%%%%%%%%%%%%%%%%%%%%%%%%%%%%%%%%%

%----------------------------------------------------------------------------------------
%	PACKAGES AND THEMES
%----------------------------------------------------------------------------------------

\documentclass{beamer}

\usepackage[T1]{fontenc}
\usepackage[polish]{babel}
\usepackage[utf8]{inputenc}
\usepackage{lmodern}
\usepackage{amsmath}

\selectlanguage{polish}

\mode<presentation> {
	\usetheme{Copenhagen}
}

\usepackage{graphicx} % Allows including images
\usepackage{booktabs} % Allows the use of \toprule, \midrule and \bottomrule in tables


% Custom macros

\newcommand{\red}[1]{
	{ \color{red}{#1} }
}

\newcommand{\score}[2]{
	\stackrel
	{\red{#1}}
	{#2}
}

\definecolor{links}{HTML}{2A1B81}
\hypersetup{colorlinks,linkcolor=,urlcolor=links}


%----------------------------------------------------------------------------------------
%	TITLE PAGE
%----------------------------------------------------------------------------------------

\title
[System obliczący wyniki wyborów]
{System obliczający wyniki wyborów dla uogólnienia systemu k-Borda}

\author
[T. Kasprzyk, D. Ogiela, J. Stępak]
{Tomasz Kasprzyk, \ Daniel Ogiela, \ Jakub Stępak}

\institute
[AGH]
{
Akademia Górniczo-Hutnicza

Wydział Informatyki, Elektroniki i Telekomunikacji

Katedra Informatyki 
\newline \newline
Projekt realizowany pod opieką \\dr. hab. inż. Piotra Faliszewskiego

}
\date{4 kwietnia 2016}

%----------------------------------------------------------------------------------------

\begin{document}

\frame{\titlepage}

\begin{frame}
\frametitle{Plan}
\tableofcontents
\end{frame}

%----------------------------------------------------------------------------------------
%	PRESENTATION SLIDES
%----------------------------------------------------------------------------------------

%------------------------------------------------
\section{Opis problemu}
%------------------------------------------------

\subsection{Wybory}

\begin{frame}
\frametitle{Wybory}
Naszym zadaniem jest obliczenie wyników wyborów.
Przez wybory możemy rozumieć zarówno „tradycyjne” wybory np. parlamentarne,
ale także, na przykład, problem wyboru odpowiednich tytułów filmowych do
systemów rozrywkowych samolotu, aby jak najwięcej pasażerów mogło wybrać
coś dla siebie.
\end{frame}

%------------------------------------------------


\begin{frame}
\frametitle{Wybory formalnie}
Wybory to para $E = (C, V)$,
gdzie $C = \{ c_1, c_2, \ldots, c_m \}$ to zbiór kandydatów,
a $V = (v_1, v_2, \ldots, v_n)$ to ciąg wyborców.
Każdy wyborca jest opisany przez swoje $preferencje$,
które są ciągiem kandydatów w porządku od najbardziej pożądanego
przez danego wyborcę do najmniej pożądanego. Rozpatrujemy wybory
komitetów, więc otrzymujemy też liczbę $k$ będącą wielkością wybieranego komitetu.
\end{frame}

%------------------------------------------------

\begin{frame}
\frametitle{Przykładowe wybory}
\begin{exampleblock}{Przykładowe wybory}


$ C = \{a, b, c, d, e, \ldots \} $ \\
$ V = (v_1, v_2, v_3, \ldots, v_n) $ \\ ~ \\
$ v_1: a > b > c > d > e > \ldots $ \\
$ v_2: c > d > a > b > f > \ldots $ \\
$ v_3: \ldots $ \\
$ \vdots $ \\
$ v_n: f > g > e > b > a > \ldots $ \\ ~ \\
$ k = 20 $


\end{exampleblock}
\end{frame}

%------------------------------------------------
\subsection{Scoring rules}

\begin{frame}
\frametitle{Metoda Bordy}

Metoda Bordy -
niech $v$ będzie głosem nad zbiorem kandydatów $C$.
Wynik wg Bordy kandydata $c \in C$ w $v$ jest równy
$ ||C|| - pos_v(c) $.
Wynik $c$ w wyborach jest sumą wyników $c$ dla każdego wyborcy.

\begin{exampleblock}{Wynik metodą Bordy}
$$
\beta(i) = m-i, \quad gdzie \ m = ||C||
$$
\end{exampleblock}

\begin{exampleblock}{Preferencje wyborcy}
$$ 
v_1: \score{m-1}{a} > \score{m-2}{b} > \score{m-3}{c} > \score{m-4}{d} > \ldots
$$
\end{exampleblock}

\end{frame}

%------------------------------------------------

\begin{frame}
\frametitle{Committee scoring rules}

\begin{exampleblock}{Preferencje wyborcy}
$$
v_3: \score{m-1}{g} > \score{m-2}{d} > \score{m-3}{e} > \score{m-4}{b} > \score{m-5}{f} > \score{m-6}{a} > \ldots
$$
\end{exampleblock}

Dla wybranego komitetu $W = {a, b, e, f}$ i danego
wyborcy $v_i$ definiujemy ciąg $ pos_i(W) $ jako
posortowany ciąg pozycji, jakie zajmują kandydaci w preferencjach wyborcy:

$$pos_{v_3}(W) = (3, 4, 5, 6)$$

\begin{exampleblock}{Wartość komitetu dla wyborcy}
$$
f(i_1, \ldots, i_k)
$$
\end{exampleblock}

\end{frame}

%------------------------------------------------

\begin{frame}
\frametitle{Committee scoring rules}

Istnieją systemy wyborcze: \\ ~ \\

\begin{tabular}{ll}
k-Borda: 			& $f_{kB}(i_1, \ldots, i_k) = \beta(i_1) + \ldots + \beta(i_k)$\\
Chamberlin–Courant:  & $f_{CC}(i_1, \ldots, i_k) = \beta(i_1)$
\end{tabular}

\end{frame}

%------------------------------------------------

\begin{frame}
\frametitle{Norma $\ell_p$}

\begin{block}{Norma $\ell_p$}
$$
\ell_p(x_1, x_2, \ldots, x_n ) = \sqrt[p]{ x_1^p + x_2^p + \ldots + x_n^p }
$$
\end{block}

\begin{exampleblock}{}
$$
\begin{aligned}
\ell_1 	\	 &\equiv 	\ \ +  \\
\ell_\infty 	\  &\equiv 	\ max 
\end{aligned}
$$
\end{exampleblock}

\end{frame}

%------------------------------------------------

\begin{frame}
\frametitle{System $\ell_p-Borda$}

Rozważmy więc system definiowany następująco:

\begin{exampleblock}{System $\ell_p-Borda$}
$$
f_{\ell_p}(i_1, i_2, \ldots, i_k ) = \ell_p( \beta(i_1) + \beta(i_2) + \ldots + \beta(i_k) )
$$
\end{exampleblock}

\begin{exampleblock}{}
$$
\begin{aligned}
f_{\ell_1} 	\	 &\equiv 	f_{k-Borda}  \\
f_{\ell_\infty} 	\  &\equiv 	\ f_{Chamberlin-Courant} 
\end{aligned}
$$
\end{exampleblock}

\end{frame}

%------------------------------------------------
\section{Projekt inżynierski}

\subsection{Nasze zadanie}

\begin{frame}
\frametitle{Nasze zadanie}
\begin{itemize}
\item Obliczanie wyników wyborów w systemie $\ell_p-Borda$ jest czasochłonne.
\item Naszym zadaniem jest znalezienie algorytmu, który pozwoli szybko przybliżać wyniki wyborów.
\item Do systemu należy dostarczyć pewien interfejs, który będzie pozwalał na wprowadzenie preferencji z otwartych baz (jak \href{http://www.preflib.org/}{PrefLib}) lub generował rozkład wyborców i kandydatów np. w\,oparciu o rozkład naturalny.
\end{itemize}
\end{frame}

%------------------------------------------------

\subsection{Plan pracy}

\begin{frame}
\frametitle{Plan pracy}
\begin{itemize}
\item [maj -] Implementacja brute-force.
\item [maj -] Równolegle praca nad algorytmem, który pozwoli szybko wykonywać obliczenia.
\item [lipiec -] Implementacja szybszej wersji obliczeń.
\item [lipiec -] Stworzenie interfejsu, import danych z PrefLib'a, symulowanie wyborów np. z rozkładu normalnego, etc.
\end{itemize}
\end{frame}

%------------------------------------------------


\begin{frame}
\Huge{\centerline{Dziękujemy za uwagę}}
\end{frame}

%----------------------------------------------------------------------------------------

\end{document} 
